\thispagestyle{plain}

\section{ESTRUCTURA CAPITULAR PROPUESTA}

\begin{enumerate}	
	\item Aspectos generales
	
	\begin{enumerate}
		\item Objetivo general
		\item Objetivos espec�ficos
		\item Sobre SIGCSA
		\item Antecedentes de la empresa
		
	\end{enumerate}
	
	\item Marco Te�rico
	
	\begin{enumerate}	
		\item Normas de calidad
		\item Tecnolog�a existente
		
		\begin{enumerate}
			\item Smartphone (Android)
			\item Arduino
			\item Sensores en general
			\item Term�metros
			\item Impresoras 3D
			\item Equipos utilizados para la calibraci�n de term�metros.
			
			\item Componentes electronicos y Arduino.
			
			\item Principios y buenas practicas de usabilidad e interfaces m�viles.
		\end{enumerate}
		
		\item Lenguajes de programaci�n
		\item Bases de datos
		\item HCI
		
		\begin{enumerate}
			\item Metodolog�as
			\item Normas
			\item Evaluaci�n
		\end{enumerate}
		
	\end{enumerate}

	\item An�lisis de procesos 
	
	\begin{enumerate}
		\item Proceso operativo SIGCSA-PO-25
		\item Proceso operativo SIGCSA-PO-14
	\end{enumerate}

\thispagestyle{empty}
\newpage
		
	\item Dise�o y Desarrollo del prototipo
	
	\begin{enumerate}
		\item Prototipo de circuito del term�metro.
		\item Desarrollo de firmware del Arduino.
		\item Dise�o de interfaz para smartphone.
		\item Desarrollo de aplicaci�n para smartphone.
		\item Dise�o de circuito del term�metro.
		\item Dise�o est�tico del term�metro.
	\end{enumerate}
	
	\item Evaluaci�n y an�lisis de resultados
	
	\begin{enumerate}
		\item Pruebas del sistema en campo.
		\item Comparaci�n del sistema propuesto, con respecto al sistema utilizado por SIGCSA.
	\end{enumerate}
	
	
	\item Conclusi�n
	
	\item Referencias
	
	\item Anexos
	
\end{enumerate}
