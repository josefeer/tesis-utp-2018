\subsection{Lenguaje JAVA - Android}
	\par 
		Java es uno de los lenguajes de programación más importantes y más utilizados del mundo.
		Además, ha mantenido esa distinción por muchos años. A diferencia de algunos otros
		lenguajes de programación cuya influencia ha disminuido con el paso del tiempo. Java saltó a la vanguardia de la programación de Internet con su primer
		lanzamiento. Cada versión posterior ha solidificado esa posición.
		
\clearpage
\thispagestyle{plain} 

	\par \noindent
		Hoy, sigue siendo el primero
		y la mejor opción para desarrollar aplicaciones basadas en web. También es un poderoso lenguaje de programación de propósito general adecuado para una amplia variedad de propósitos.
		Gran parte del mundo moderno se ejecuta en código Java.
		
	\par \noindent
		El lenguaje de programación Java fue originalmente desarrollado por James Gosling, de Sun Microsystems (la cual fue adquirida por la compañía Oracle), y publicado en 1995 como un componente fundamental de la plataforma Java de Sun Microsystems. Su sintaxis deriva en gran medida de C y C++, pero tiene menos utilidades de bajo nivel que cualquiera de ellos. Las aplicaciones de Java son generalmente compiladas a bytecode (clase Java), que puede ejecutarse en cualquier máquina virtual Java (JVM) sin importar la arquitectura de la computadora subyacente.
		
	\par \noindent
		El desarrollo de programas o aplicaciones en android se hace habitualmente con el lenguaje de programación JAVA y el SDK (Software Development Kit, por sus siglas en inglés). El SDK de Android incluye un conjunto de herramientas de desarrollo. Comprende un depurador de código, una biblioteca, un simulador de teléfono, documentación, ejemplos de código y tutoriales.
		
	\par \noindent
		El IDE (Integrated Development Enviroment, por sus siglas en inglés) oficial es Android Studio, el cual vamos a estar utilizando en este proyecto, en donde se utiliza un editor de texto para editar archivos JAVA, XML y comandos en la terminal para crear y depurar aplicaciones.