\thispagestyle{plain}
	\section{Bases de Datos}
		\par 
			Un sistema gestor de bases de datos (SGBD) consiste en una colección de datos interelacionados y un conjunto de programas para acceder a dichos datos. La colección de datos, normalmente denominada base de datos, contiene información relevante para una empresa. El objetivo principal de un SGBD es proporcionar una forma de almacenar y recuperar la información de una base de datos de manera que sea tanto práctica como eficiente.
			
		\par \noindent
			Los sistemas de bases de datos se diseñan para gestionar grandes cantidades de información. La gestión de datos implica tanto definición de estructuras para almacenar la información como la provisión de mecanismos para la manipulación de la información. Además, los sistemas de bases de datos deben proporcionar la fiabilidad de la información almacenada, a pesar de las caídas del sistema o los intentos de acceso sin autorización. Si los datos van a hacer compartidos entre diversos usuarios, el sistema debe evitar posibles resultados anómalos.
			
		\par \noindent
			Un propósito principal de un sistema de bases de datos
			es proporcionar a los usuarios una visión abstracta de
			los datos. Es decir, el sistema esconde ciertos detalles
			de cómo los datos se almacenan y mantienen.
			
		\par \noindent
			Por debajo de la estructura de la base de datos está el
			modelo de datos: una colección de herramientas conceptuales para describir los datos, las relaciones entre
			los datos, la semántica de los datos y las restricciones
			de los datos. El modelo de datos entidad-relación es un
			modelo de datos ampliamente usado, y proporciona una
			representación gráfica conveniente para ver los datos,
			las relaciones y las restricciones. Otros modelos de datos son el modelo de datos orientado a objetos, el relacional orientado
			a objetos, el modelo de datos semiestructurados y el modelo relacional.
			
		\subsection{Modelo Entidad-Relación}
	\par 
		El modelo de datos entidad-relación (E-R) está basado en una percepción del mundo real consistente en objetos básicos llamados entidades y de relaciones entre estos objetos. Se desarrolló para facilitar el diseño de bases de datos permitiendo la especificación de un esquema de una empresa que representa una estructura lógica completa de una base de datos. El modelo de datos E-R es uno de los diferentes modelos de datos semánticos; el aspecto semántico del modelo yace en la represetación del significado de los datos. 
		
\clearpage
\thispagestyle{plain}

	\par \noindent
		El modelo E-R es extremadamente útil para hacer corresponder los significados e interacciones de las empresas del mundo real con un esquema conceptual.
		Debido a esta utilidad, muchas herramientas de diseño de bases de datos se basan en los conceptos del modelo E-R. Hay tres nociones básicas que emplea el modelo de datos E-R: conjuntos de entidades, conjuntos de relaciones y atributos.
		
	\subsubsection{Conjunto de entidades}
		\par \noindent
			Una entidad es un objeto en el mundo real que es distinguible de todos los demás objetos. Una entidad tiene un conjunto de propiedades, y los valores para algun conjunto de propiedades pueden identificar una entidad de forma unívoca. Una entidad puede ser tangible como una persona o un libro, o puede ser abstracta, como un prestamo, unas vacaciones o un concepto.
			
		\par \noindent
			Un conjunto de entidades es un conjunto de entidades del mismo tipo que comparten las mismas propiedades o atributos. Una entidad se representa mediante un conjunto de atributos. Los atributos describen propiedades que poseen cada miembro de un conjunto de entidades. La designación de un atributo para un conjunto de entidades expresa que la base de datos almacena información similar concerniente a cada entidad del conjunto de entidades; sin embargo, cada entidad puede tener su propio valor para cada atributo. Cada entidad tiene un valor para cada uno de sus atributos.
			
	\subsubsection{Conjunto de relaciones}
		\par \noindent
			Una relación es una asociación entre diferentes entidades. Un conjunto de relaciones es un conjunto de relaciones del mismo tipo. La asociación entre conjuntos de entidades se conoce como participación y la función que desempeña una entidad en una relación se llama papel de la entidad.
			Debido a que los conjuntos de entidades que participan en un conjunto de relaciones son generalmente distintos, los papeles están implícitos y no se especifican
			normalmente.
			
\clearpage
\thispagestyle{plain}
		
		\subsection{SQLite}
	\par 
		SQLite es un motor de base de datos SQL embebido, esto quiere decir que trabaja como un manejador de bases de datos local. Es usado por nombres prominentes como
		Adobe en Adobe Integrated Runtime (AIR); Airbus, en su software de vuelo; el lenguaje de programción Python viene 
		con SQLite; PHP; y muchos más. En el dominio móvil, SQLite es una opción muy popular
		en varias plataformas debido a su naturaleza liviana. Apple lo usa en el iPhone y
		Google en el sistema operativo Android.
	
	\par \noindent	
		Las ventajas de utilizar SQLite son:
		
		\begin{itemize}
			\item Cero-configuración: SQLite está diseñado de tal manera que no requiere
			archivos de configuración. No requiere pasos de instalación o configuración inicial; no tiene un
			proceso de servidor en ejecución y sin pasos de recuperación, incluso si falla. No hay servidor
			y está directamente integrado en nuestra aplicación. Además, no hay administrador
			requerido para crear o mantener una instancia de la base de datos, o establecer permisos para los usuarios. En breve,
			esta es una verdadera base de datos sin DBA (Data Base Administrator, por sus siglas en inglés).
			
			\item Sin Derechos de Autor: El código fuente
			de SQLite es de dominio público.
			
			\item Multiplataforma: Los archivos de base de datos de un sistema se pueden mover a un sistema que ejecuta un
			arquitectura diferente sin ningún tipo de molestia. Esto es posible porque el archivo de la base de datos se encuentra en un formato binario y todas las máquinas usan el mismo formato.
			
			\item Compacto: Una base de datos SQLite es un único archivo de disco ordinario; viene sin un
			servidor y está diseñado para ser ligero y simple. Estos atributos conducen a una muy
			motor de base de datos liviano.
			
			\item Lenguaje SQL: El lenguaje utilizado en SQLite es el lenguaje SQL, el cual es ampliamente utilizado por desarrolladores y de uso común en bases de datos relacionales.
			
		\end{itemize}
		
		