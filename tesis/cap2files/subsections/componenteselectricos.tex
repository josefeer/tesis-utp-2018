\subsection{Componentes Electronicos}
	\subsubsection{Resistencia} 
		\par 
			Son usados para establecer corrientes de operación y niveles de señal. Resistencias se utilizan en los circuitos de alimentación para reducir los voltajes al disipar la potencia, medir las corrientes y descargar los condensadores después de que se desconecta la energía. Una resistencia está hecha de elementos conductores (carbono, o una película delgada de metal o carbono, o un cable de baja conductividad), con un cable o contactos en cada extremo. Se caracteriza por su resistencia:
			$$R = V/I;$$
		
		\par \noindent
			$R$ está en ohmios para $V$ en voltios e $I$ en amperios. Esto es conocido
			como la ley de Ohm.
		
		\par \noindent
			Resistencias típicas de las más utilizadas
			tipo entrar valores de 1 ohmio (1) a alrededor de 10 megaohmios (10M).Las resistencias también se caracterizan por la potencia que tienen puede disiparse con seguridad (los más comúnmente usados son con una calificación de 1/4 o 1/8 W),su tamaño físico y otros parámetros tales como tolerancia (precisión), coeficiente de temperatura, ruido, coeficiente de voltaje (la medida en que R depende de V aplicado), estabilidad con el tiempo, inductancia, etc.
		
	\subsubsection{Capacitor}
		\par 
			Un capacitor (el nombre antiguo era
			condensador) es un dispositivo que tiene dos cables que sobresalen y tiene la propiedad
			$$Q = CV$$ 
			
		\par \noindent
			Su forma básica es un par de placas de metal espaciadas de cerca, separados por algún material aislante. Un condensador de $C$ faradios con $V$ voltios a través de sus terminales tiene $Q$ coulombs de carga almacenada en una placa y $-Q$ en la otra. La capacitancia es proporcional al área e inversamente proporcional al espacio.
		
		\par \noindent
			Los capacitores esencialmente almacenan energía, pero es principalmente utilizado en corriente directa para filtrar picos de voltage provenientes de la fuente de la fuente de poder.
		
		\newpage
		\thispagestyle{plain}
		
	\subsubsection{Inductor} 
		\par
			Están estrechamente relacionados con condensadores: la tasa de cambio de corriente en un inductor
			es proporcional al voltaje aplicado a través de él (para un condensador es al revés, la tasa de cambio de voltaje es proporcional a la corriente a través de él). La ecuación de definición para un inductor es:
			$$V = L\frac{dl}{dt}$$
		
		\par \noindent
			donde $L$ se llama inductancia y se mide en henrios
			(o $mH$, $pH$, $nH$, etc.). Poniendo un voltaje constante a través de un
			inductor hace que la corriente se eleve como una rampa (en comparación con un condensador, en el que una corriente constante causa el voltaje subir como una rampa); l $V$ a través de 1 $H$ produce una corriente que aumenta a 1 amperio por segundo.
	
		\par \noindent
			Al igual que con los capacitores, la energía invertida en el aumento de la corriente en un inductor se almacena internamente, aquí en la forma de campos magnéticos.
			
		\par \noindent
			El inductor básico es una bobina, que puede ser solo una vuelta con uno o más giros de alambre; o puede ser una bobina con algo de longitud, conocido como solenoide. Las variaciones incluyen bobinas hecha en varios materiales centrales, siendo el más popular hierro (o aleaciones de hierro, laminaciones o polvo) y ferrita (un material magnetico gris, no conductor, gris). Un inductor realmente es lo opuesto de un capacitor.