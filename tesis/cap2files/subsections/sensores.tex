\subsection{Sensores en General}
	\par
		Un sensor es un dispositivo capaz de detectar magnitudes físicas o químicas, llamadas variables de instrumentación, y transformarlas en variables eléctricas.
		\begin{itemize}
			\item Las variables de instrumentación pueden ser por ejemplo: temperatura, intensidad lumínica, distancia, aceleración, inclinación, desplazamiento, presión, fuerza, torsión, humedad, movimiento, pH, etc.
			
			\item Una variable eléctrica puede ser una resistencia eléctrica (como en una RTD), una capacidad eléctrica (como en un sensor de humedad o un sensor capacitivo), una tensión eléctrica (como en un termopar), una corriente eléctrica (como en un fototransistor), etc.
		\end{itemize}

		\noindent
		Los sensores se pueden clasificar en función de los datos de salida en: digitales y analógicos.
	
\subsubsection{Características de los Sensores}

	\begin{itemize}
		\item Rango de medida: dominio en la magnitud medida en el que puede aplicarse el sensor.
		
		\item Precisión: es el error de medida máximo esperado.
		
		\item Offset o desviación de cero:  valor de la variable de salida cuando la variable de entrada es nula. Si el rango de medida no llega a valores nulos de la variable de entrada, habitualmente se establece otro punto de referencia para definir el offset.
		
		\item Linealidad o correlación lineal.
		
		\item Sensibilidad de un sensor: suponiendo que es de entrada a salida y la variación de la magnitud de entrada.
		
		\item Resolución: mínima variación de la magnitud de entrada que puede detectarse a la salida.
		
		\item Rapidez de respuesta: puede ser un tiempo fijo o depender de cuánto varíe la magnitud a medir. Depende de la capacidad del sistema para seguir las variaciones de la magnitud de entrada.
		
		\newpage
		\thispagestyle{plain}
		
		\item Derivas: son otras magnitudes, aparte de la medida como magnitud de entrada, que influyen en la variable de salida. Por ejemplo, pueden ser condiciones ambientales, como la humedad, la temperatura u otras como el envejecimiento (oxidación, desgaste, etc.) del sensor.
		
		\item Repetitividad: error esperado al repetir varias veces la misma medida.
	\end{itemize}

\subsubsection{Sensores de Temperatura}
	\begin{itemize}
		\item DS18B20
		\item DHT11
	\end{itemize}