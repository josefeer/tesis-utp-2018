\subsection{Smartphone}
			\par 
				El teléfono inteligente (smartphone en inglés) es un tipo de ordenador de bolsillo que combina los elementos de una tablet con los de un teléfono celular. Sobre una plataforma informática móvil, con mayor capacidad de almacenar datos y realizar actividades, semejante a la de una minicomputadora, y con una mayor conectividad que un teléfono móvil convencional. El término inteligente, que se utiliza con fines comerciales, hace referencia a la capacidad de usarse como un computador de bolsillo, y llega incluso a reemplazar a una computadora personal en algunos casos.		
			\par \noindent
				Generalmente, los teléfonos con pantallas táctiles son los llamados teléfonos inteligentes, pero el soporte completo al correo electrónico parece ser una característica indispensable encontrada en todos los modelos existentes y anunciados desde 2007. Casi todos los teléfonos inteligentes también permiten al usuario instalar programas adicionales, habitualmente incluso desde terceros, hecho que dota a estos teléfonos de muchísimas aplicaciones en diferentes terrenos; sin embargo, algunos teléfonos son calificados como inteligentes aun cuando no tienen esa característica.			
			\par \noindent
				Entre otros rasgos comunes está la función multitarea, el acceso a Internet vía Wifi o redes 2G, 3G o 4G, función multimedia (cámara y reproductor de videos/mp3), a los programas de agenda, administración de contactos, acelerómetros, GPS y algunos programas de navegación, así como ocasionalmente la habilidad de y leer documentos de negocios en variedad de formatos como PDF y Microsoft Office.
			\par \noindent
				En la actualidad, el diseño de los smartphones es muy similar entre ellos: rectángular, con una o dos cámaras (tanto frontal como posterior) y algunos botones (generalmente 3, +volumen, -volumen y un botón para el bloqueo/encendido/apagado del dispositivo) ya que es totalmente táctil.
			\par \noindent
				Los sistemas operativos móviles más frecuentes utilizados por los teléfonos inteligentes son Android (de Google), iOS (de Apple) y Windows 10 (de Microsoft).