\subsection{Termómetros}
	\par 
		El termómetro es un instrumento de medición de temperatura. Desde su invención ha evolucionado mucho, principalmente a partir del desarrollo de los termómetros electrónicos digitales.
	\par \noindent
		Inicialmente se fabricaron aprovechando el fenómeno de la dilatación, por lo que se prefería el uso de materiales con elevado coeficiente de dilatación, de modo que, al aumentar la temperatura, su estiramiento era fácilmente visible. La sustancia que se utilizaba más frecuentemente en este tipo de termómetros ha sido el mercurio, encerrado en un tubo de vidrio que incorporaba una escala graduada, pero también alcoholes coloreados en termómetros grandes.
	\par \noindent
		El creador del primer termoscopio fue Galileo Galilei; este podría considerarse el predecesor del termómetro. Consistía en un tubo de vidrio terminado en una esfera cerrada; el extremo abierto se sumergía boca abajo dentro de una mezcla de alcohol y agua, mientras la esfera quedaba en la parte superior. Al calentar el líquido, este subía por el tubo.
		
	\subsubsection{Escalas de temperatura}
		\par 
			La escala más usada en la mayoría de los países del mundo es la Celsius (\textdegree{}C) en honor a Anders Celsius (1701-1744) que se llamó centígrado hasta 1948. En esta escala, el cero (0 \textdegree{}C) y los cien (100 \textdegree{}C) grados corresponden respectivamente a los puntos de congelación y de ebullición del agua, ambos a la presión de 1 atmósfera.
			
		\par \noindent 
		Otras escalas termométricas son:
			
		\begin{itemize}
			\item Fahrenheit (\textdegree{}F):  propuesta por Daniel Gabriel Fahrenheit en la revista Philosophical Transactions (Londres, 33, 78, 1724). El grado Fahrenheit es la unidad de temperatura en el sistema anglosajón de unidades, utilizado principalmente en Estados Unidos.
			
			\item Kelvin o temperatura absoluta, es la escala de temperatura del Sistema Internacional de Unidades. Aunque la magnitud de una unidad Kelvin (K) coincide con un grado Celsius (\textdegree{}C), el cero se ha fijado en el cero absoluto a -273,15 \textdegree{}C y es inalcanzable según el tercer principio de la termodinámica.
		\end{itemize}
	
		\newpage
		\thispagestyle{plain}
	
	\subsubsection{Tipos de termómetros}
	
		\begin{itemize}
			\item Termómetro de mercurio: es un tubo de vidrio sellado que contiene mercurio, cuyo volumen cambia con la temperatura de manera uniforme. Este cambio de volumen se aprecia en una escala graduada. El termómetro de mercurio fue inventado por Gabriel Fahrenheit en el año 1714.
			
			\item Pirómetros:  termómetros para altas temperaturas, se utilizan en fundiciones, fábricas de vidrio, hornos para cocción de cerámica, etc. Existen varios tipos según su principio de funcionamiento:
			
			\begin{itemize}
				\item Pirómetro óptico: se basan en la ley de Wien de distribución de la radiación térmica, según la cual, el color de la radiación varía con la temperatura. El color de la radiación de la superficie a medir se compara con el color emitido por un filamento que se ajusta con un reostato calibrado. Se utilizan para medir temperaturas elevadas, desde 700 °C hasta 3.200 °C, a las cuales se irradia suficiente energía en el espectro visible para permitir la medición óptica.
				
				\item Pirómetro de radiación total: se fundamentan en la ley de Stefan-Boltzmann, según la cual, la intensidad de energía emitida por un cuerpo negro es proporcional a la cuarta potencia de su temperatura absoluta.
				
				\item Pirómetro infrarojos: captan la radiación infrarroja, filtrada por una lente, mediante un sensor fotorresistivo, dando lugar a una corriente eléctrica a partir de la cual un circuito electrónico calcula la temperatura. Pueden medir desde temperaturas inferiores a 0 \textdegree{}C hasta valores superiores a 2.000 \textdegree{}C.
				
				\item Pirómetro fotoelectrónico: se basan en el efecto fotoeléctrico, por el cual se liberan electrones de semiconductores cristalinos cuando incide sobre ellos la radiación térmica.
			\end{itemize}
			
			\item Termómetro de lámina bimetálica: formado por dos láminas de metales de coeficientes de dilatación muy distintos y arrollados dejando el coeficiente más alto en el interior. Se utiliza sobre todo como sensor de temperatura en el termohigrógrafo.
			
			\item Termómetro de gas: pueden ser a presión constante o a volumen constante. Este tipo de termómetros son muy exactos y generalmente son utilizados para la calibración de otros termómetros.
			
			\item Termómetro de resistencia: consiste en un alambre de algún metal (como el platino) cuya resistencia eléctrica cambia cuando varía la temperatura.
			
			\newpage
			\thispagestyle{plain}
			
			\item Termopar: un termopar o termocupla es un dispositivo utilizado para medir temperaturas basado en la fuerza electromotriz que se genera al calentar la soldadura de dos metales distintos.
			
			\item Termistor:  es un dispositivo que varía su resistencia eléctrica en función de la temperatura. Algunos termómetros hacen uso de circuitos integrados que contienen un termistor, como el LM35.
			
			\item Termómetros digitales: son aquellos que, valiéndose de dispositivos transductores como los mencionados, utilizan luego circuitos electrónicos para convertir en números las pequeñas variaciones de tensión obtenidas, mostrando finalmente la temperatura en un visualizador. Una de sus principales ventajas es que por no utilizar mercurio no contaminan el medio ambiente cuando son desechados.
			
			\item Termómetros clinicos: son los utilizados para medir la temperatura corporal. Los hay tradicionales de mercurio y digitales, teniendo estos últimos algunas ventajas adicionales como su fácil lectura, respuesta rápida, memoria y en algunos modelos alarma vibrante.
			
			
			
			\item Termógrafo: El termógrafo es un termómetro acoplado a un dispositivo capaz de registrar, gráfica o digitalmente, la temperatura medida en forma continua o a intervalos de tiempo determinado.
			
		\end{itemize}