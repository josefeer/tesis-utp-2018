\subsection{Modelo Entidad-Relación}
	\par 
		El modelo de datos entidad-relación (E-R) está basado en una percepción del mundo real consistente en objetos básicos llamados entidades y de relaciones entre estos objetos. Se desarrolló para facilitar el diseño de bases de datos permitiendo la especificación de un esquema de una empresa que representa una estructura lógica completa de una base de datos. El modelo de datos E-R es uno de los diferentes modelos de datos semánticos; el aspecto semántico del modelo yace en la represetación del significado de los datos. 
		
\clearpage
\thispagestyle{plain}

	\par \noindent
		El modelo E-R es extremadamente útil para hacer corresponder los significados e interacciones de las empresas del mundo real con un esquema conceptual.
		Debido a esta utilidad, muchas herramientas de diseño de bases de datos se basan en los conceptos del modelo E-R. Hay tres nociones básicas que emplea el modelo de datos E-R: conjuntos de entidades, conjuntos de relaciones y atributos.
		
	\subsubsection{Conjunto de entidades}
		\par \noindent
			Una entidad es un objeto en el mundo real que es distinguible de todos los demás objetos. Una entidad tiene un conjunto de propiedades, y los valores para algun conjunto de propiedades pueden identificar una entidad de forma unívoca. Una entidad puede ser tangible como una persona o un libro, o puede ser abstracta, como un prestamo, unas vacaciones o un concepto.
			
		\par \noindent
			Un conjunto de entidades es un conjunto de entidades del mismo tipo que comparten las mismas propiedades o atributos. Una entidad se representa mediante un conjunto de atributos. Los atributos describen propiedades que poseen cada miembro de un conjunto de entidades. La designación de un atributo para un conjunto de entidades expresa que la base de datos almacena información similar concerniente a cada entidad del conjunto de entidades; sin embargo, cada entidad puede tener su propio valor para cada atributo. Cada entidad tiene un valor para cada uno de sus atributos.
			
	\subsubsection{Conjunto de relaciones}
		\par \noindent
			Una relación es una asociación entre diferentes entidades. Un conjunto de relaciones es un conjunto de relaciones del mismo tipo. La asociación entre conjuntos de entidades se conoce como participación y la función que desempeña una entidad en una relación se llama papel de la entidad.
			Debido a que los conjuntos de entidades que participan en un conjunto de relaciones son generalmente distintos, los papeles están implícitos y no se especifican
			normalmente.
			
\clearpage
\thispagestyle{plain}