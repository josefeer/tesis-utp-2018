\subsection{Selección de Componentes}
\par 
Al momento de seleccionar los componentes se tomó en cuenta factores como dimensión, flexibilidad de integración y costo. Los elementos utilizados para la construcción del modelo son:

\begin{itemize}
	\item Arduino nano (Versión 3.0): Utiliza el microcontrolador ATMEGA328P, el cual es el mismo al Arduino Uno, ver figura 1.5, por lo que la documentación es muy amplia para este tipo de placa. Es compacto y puede ser utilizado fácilmente en un breadboard o soldado a una placa, ver figura 1.6. Brinda una memoria de 32KB suficiente para un código amplio, 14 pines digitales, menos 2 pines que son utilizados para transmisión(TX) y recepción (RX), y 8 pines análogos, 6 pueden ser utilizados como digitales, brindando un total de 18 pines programables para actuar como salidas o entradas de información.
	
	\item DS18B20: Como se ha mencionado previamente en el marco teórico este trabajo, el sensor viene en una forma de sonda, ver figura 1.18. Es muy versátil y tiene una resolución y error de medición aceptable para mediciones industriales y se puede encontrar en longitudes de hasta 1 metro. Su costo es mínimo, aproximadamente de 3 dólares por sensor. 
	
	\item LCD TFT 2.8" ILI9341: A pesar de tener la que mayor taza de consumo de corriente entre todas las piezas de su categoría. Ofrece una gran tamaño para poder visualizar de manera eficiente la información. Puede ser alimentada por la salida de 5V de un Arduino nano y cuenta con características para ser utilizada como una pantalla táctil. El LCD es el elemento que brinda mayor costo a nuestro prototipo, pero es uno de los más importantes.
	
	\item Modulo nRF24L01+: Cuenta con capacidades de comunicación a través de radiofrecuencia, es sencillo de comunicar con cualquier placa Arduino y posee una tasa de consumo de corriente mínima. El nRF24L01+ nos permitió comunicar múltiples Arduino entre sí para poder enviar la información de los sensores de temperatura.
	
	\item Modulo Bluetooth (HC-05): Pieza que utiliza la tecnología bluetooth para transmisión de información de manera inalámbrica. No es necesario utilizarlo en todos los prototipos; ya que, la información es enviada a través de radiofrecuencia. Sin embargo, un solo arquetipo es el encargado de enviar la información de todos los demás al teléfono inteligente y ahí su importancia en este proyecto. 
	
\end{itemize}

\par \noindent
Entre los componentes pasivos utilizados son: 

\begin{itemize}
	\item Capacitores Electrolíticos de 10uF
	
	\item Capacitores de Cerámica de 1nF
	
	\item Resistencias de 200, 10K, 100K, 200L y 4.7K ohmios
\end{itemize}

\par \noindent
Los capacitores se implementaron para brindar energía eléctrica estable a los componentes, las resistencias para comunicar los distintos componentes como pantalla LCD, sensor y módulos con el Arduino. Por último, se utilizó un interruptor para brindar alimentación de la fuente de poder a nuestro prototipo y una entrada Jack de 3.5 mm.

\par \noindent
Ahora se tiene claro porque elegimos ciertos componentes, se comienza a probar cada uno de ellos de manera individual.
