\subsection{Servicios en Segundo Plano}

\par 
Según la documentación oficial de android un servicio es un punto de entrada de propósito general para mantener una aplicación ejecutándose en segundo plano por todo tipo de razones. Es un componente que se ejecuta en segundo plano para realizar operaciones de larga ejecución o para realizar trabajos en procesos remotos. Un servicio no proporciona una interfaz de usuario. Por ejemplo, un servicio puede reproducir música en segundo plano mientras el usuario está en una aplicación diferente, o puede obtener datos a través de la red sin bloquear la interacción del usuario con una actividad. Otro componente, como una actividad, puede iniciar el servicio y permitir que se ejecute o se vincule con él para interactuar con él. En realidad, hay dos servicios semánticos muy distintos que le dicen al sistema cómo administrar una aplicación: los servicios iniciados le dicen al sistema que los mantenga funcionando hasta que se complete su trabajo. Esto podría ser para sincronizar algunos datos en segundo plano o reproducir música incluso después de que el usuario abandone la aplicación.\cite{androidapp}

\par \noindent
Mencionado anteriormente nuestra aplicación cuenta con dos servicios que se ejecutan en el segundo plano uno para mantener una comunicación bluetooth entre el prototipo y la aplicación. El otro para capturar las medidas de temperatura de nuestro prototipo. 

\subsubsection{Servicio Bluetooth}

\par \noindent
Cuando se esta desarrollando una aplicación en Android; ya se encuentran disponible un paquete genérico para utilizar la antena bluetooth de la mayoría de los smartphone. Este paquete "android.bluetooth" contiene clases para ciertas tareas individuales para el manejo de la antena. Las clases utilizadas del paquete "android.bluetooth" son:

\begin{itemize}
	\item BluetoothAdapter
	
	\item BluetoothDevice
	
	\item BluetoothSocket
\end{itemize}

\par \noindent
Las clases anteriormente mencionadas contienen funciones para obtener información de un dispositivo bluetooth que se encuentra conectado a nuestro smartphone. Otra función importante es que nos permiten instanciar objetos con información del dispositivo al que nos encontramos conectado, objetos con información de nuestra antenna e información del "socket" si hay una conexión activa. 

\par \noindent
Un socket es un objeto de software que actúa como un punto final que establece un enlace bidireccional de comunicación de red entre un lado del servidor y un programa del lado del cliente.\cite{socket}

\par \noindent
En Java, lenguaje utilizado en el desarrollo de aplicaciones en android, las clases de socket representan la comunicación entre los programas del cliente y del servidor. Las clases de socket manejan la comunicación del lado del cliente, y las clases de socket del servidor manejan la comunicación del lado del servidor.\cite{socket}

\par \noindent
Teniendo claro lo necesario para desarrollar el servicio, debemos desarrollar 3 clases que se comunicaran entre si.

\begin{itemize}
	\item BTAsynk: Es la clase que se encargara de establecer una comunicación con el dispositivo bluetooth deseado. Se encarga de guardar información de la conexión en memoria y valida si la conexión es exitosa.
	
	\item BTsocketHandler: Es una clase publica pero sus atributos y métodos son estáticos. Esto significa que los valores asignados y retornados de esta clase son compartidas en todas las instancias de ella. Esto es importante porque garantiza que no hallan fugas de memoria en nuestra aplicación. En ella se guarda la información obtenida por BTAsynk para ser utilizadas después.
	
	\item BTservice: Una vez BTAsynk ha establecido una conexión correctamente, esta clase hace llamados a atributos de BTsocketHandler para validar que seguimos conectados al dispositivo bluetooth y envía esa información del segundo plano al primer plano de nuestra aplicación. Esto es gracias a que BTservice hereda atribumétodos de la clase IntentService.
	
\end{itemize}

\par \noindent
Segun la documentación oficial de android la clase IntentService es una clase base para servicios que manejan solicitudes asíncronas (expresadas como Intents) a pedido. Los clientes envían solicitudes a través de llamadas Context.startService (Intent); el servicio se inicia según sea necesario, maneja cada intención a su vez utilizando un hilo de trabajo, y se detiene cuando se queda sin trabajo.\cite{intentservice}

\par \noindent
Este patrón de "procesador de cola de trabajo" se usa comúnmente para descargar tareas del hilo principal de una aplicación. La clase IntentService existe para simplificar este patrón y cuidar la mecánica de la aplicación. \cite{intentservice}

\par \noindent
En conclusión la clase BTAsynk es la encargada de establecer una comunicación bluetooth con el prototipo. Si la conexión es exitosa entonces BTAsynk envía información del dispositivo y socket activo a la clase BTSocketHandler y por ultimo inicia la clase BTService. La clase BTService entra en un ciclo while donde transformará la medidas de temperatura enviadas por el prototipo a un objeto JSON y el objeto JSON es enviado a la clase BTsocketHandler. Adicional durante este ciclo BTService estara enviando un mensaje a la actividad principal indicando que la conexión esta activa. En la actividad principal mientras BTService se encuentre enviando el mensaje, el icono del boton bluetooth sera un gancho como en la imagen 4.24. Cuando el mensaje ya no sea recibido entonces enviara la notificación al usuario que la conexión con el prototipo se a caido como en la imagen 4.22. Los códigos para las clases anteriormente mencionadas se encuentran en el Anexo 2.

\subsubsection{Servicio de Registro de Temperatura}

\par 
El Servicio de Registro de Temperatura es el encargado de convertir el objeto JSON de la clase BTSocketHandler a consultas de SQL para guardarlas en la base de datos SQLite de nuestro dispositivo.

\par \noindent
SQLite es una libreria en proceso que implementa un motor de base de datos SQL transaccional independiente, sin servidor y de configuración cero. El código para SQLite es de dominio público y, por lo tanto, es gratuito para cualquier uso, comercial o privado. SQLite es la base de datos más implementada del mundo con más aplicaciones de las que podemos contar, incluidos varios proyectos de alto perfil.SQLite es un motor de base de datos SQL incorporado. A diferencia de la mayoría de las otras bases de datos SQL, SQLite no tiene un proceso de servidor por separado. SQLite lee y escribe directamente en archivos de disco ordinarios. Una base de datos SQL completa con múltiples tablas, índices, disparadores y vistas, está contenida en un solo archivo de disco.\cite{sqlite}

\par \noindent
Anteriormente se había mecionado que la clase BTSocketHandler es estática para que sus instancias compartan la misma información. Esto es esencial debido a que para realizar inserciones en la base de datos en tiempo real es necesario que el servicio bluetooth y el servicio registro de temperatura se encuentren trabajando simultaneamente en el segundo plano, sin afectar el rendimiento de la aplicación y la clase BTSocketHandler actuando como un puente entre ambos servicios.

\par \noindent
En android no hay clases para crear o manejar una base de datos; sin embargo, el paquete "android.database" nos brinda las clases necesarias para nosotros desarrollar una sola clase que se encargue del manejo de la base de datos de nuestra aplicación. Vamos a necesitar del paquete "android.database" las siguientes clases:

\begin{itemize}
	
	\item android.database.Cursor
	
	\item android.database.sqlite.SQLiteDatabase
	
	\item android.database.sqlite.SQLiteOpenHelper
	
\end{itemize} 

\par \noindent
Con estas clases podemos desarrollar una clase que es la utilizaremos en nuestra aplicación para crear la base de datos y realizar las diferentes consultas e inserciones a la base de datos. El Código de esta clase llamada "DatabaseHelper" se encuentra en el Anexo 3.

\par \noindent
Para iniciar el servicio basta con presionar el botón "INICIAR ENSAYO" del fragmento añadir. Automáticamente el fragmento añadir envía los datos de su interfaz como los campos de textos y los parámetros seleccionados por el usuario. Luego el servicio obtiene esos valores y creamos una instancia de la clase "DatabaseHelper" la cual si el fichero de la base de datos no se ha creado la crea automáticamente. 

\par \noindent
Los valores del fragmento añadir son ordenados para realizar la primera inserción en la base de datos. Si esta primera inserción es valida entonces preguntamos el valor del tiempo de estabilización seleccionado en el fragmento añadir. Si el valor es diferente de 0 entonces utilizamos el comando "SystemClock.sleep" para detener la ejecución del servicio por el tiempo que se solicito.

\par \noindent
Luego dependiendo del tiempo de tipo de ensayo (Isotermo 1, Isotermo 2 o Calibración con Patrón) entonces seleccionamos la tabla y los valores a insertar. Una vez seleccionados el servicio entra en un ciclo for donde se estara llamando el objeto JSON de la clase BTsocketHandler para realizar las inserciones correspondientes, el ciclo for se ejecutara por el tiempo que estipulo el usuario y cada vez que se realice un ciclo se envía el porcentaje del ensayo hasta el momento al "progressbar" del fragmento añadir.

\par \noindent
Por ultimo se valida que el usuario no presiono el botón cancelar y se actualiza la primera inserción realizada en el servicio.

\par \noindent
Ambos servicios interactuan entre si para crear validaciones al usuario y poder actualizar la base de datos correctamente. Sin embargo es necesario saber la estructura de la base de datos; ya que, no solamente es utilizada por los servicios sino también por el usuario al momento de consultar la información capturada. 

\clearpage