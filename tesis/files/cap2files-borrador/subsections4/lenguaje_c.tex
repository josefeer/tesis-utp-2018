\subsection{Lenguaje C/C++}
	\par
		C es un lenguaje de programación de propósito general que presenta economía de expresión, flujo de control moderno y estructuras de datos, y un amplio conjunto de operadores.
		C no es un lenguaje de "muy alto nivel", ni "grande", y no está especializado en ningún área particular de aplicación. Pero su ausencia de restricciones y su generalidad la hacen más conveniente y efectiva para muchas tareas que los lenguajes supuestamente más poderosos.
		
	\par \noindent
		C fue originalmente diseñado e implementado en el sistema operativo UNIX en DEC PDP-11, por Dennis Ritchie. El sistema operativo, el compilador C y, esencialmente, todos los programas de aplicaciones UNIX están escritos en C. C no está vinculado a ningún hardware o sistema en particular haciendolo fácil escribir programas que se ejecutarán sin cambios en cualquier máquina que admita C .
		
	\par \noindent
		En nuestro proyecto a pesar de que Arduino comenta que el lenguaje es basado en "Processing"; sin embargo, el lenguaje Arduino es realmente basado en C/C++.
		
	\par \noindent
		C++ es un lenguaje de programación diseñado a mediados de los años 1980 por Bjarne Stroustrup. La intención de su creación fue el extender al lenguaje de programación C mecanismos que permiten la manipulación de objetos. En ese sentido, desde el punto de vista de los lenguajes orientados a objetos, el C++ es un lenguaje híbrido.
		
		Posteriormente se añadieron facilidades de programación genérica, que se sumaron a los paradigmas de programación estructurada y programación orientada a objetos. Por esto se suele decir que el C++ es un lenguaje de programación multiparadigma. 