\subsection{Interfaz de usuario}
	\par 
		La interfaz de usuario es el espacio donde se producen las interacciones entre seres humanos y máquinas. El objetivo de esta interacción es permitir el funcionamiento y control más efectivo de la máquina desde la interacción con el humano.
		
	\par \noindent
		Las interfaces básicas de usuario son aquellas que incluyen elementos como menús, ventanas, contenido gráfico, cursor, los beeps y algunos otros sonidos que la computadora hace, y en general, todos aquellos canales por los cuales se permite la comunicación entre el ser humano y la computadora.
		
	\par \noindent
		El objetivo del diseño de una interfaz es producir una interfaz que sea fácil de usar (explicarse por sí misma), eficiente y agradable para que al operar la máquina dé el resultado deseado.
		
	\par \noindent
		Ahora que sabemos la definición de lo que es el interfaz de usuario y su función, ahora veremos características de una mala intefaz.
		
	\subsubsection{Controles de interfaz no estandarizados} 
		\par 
			Los widgets básicos de la GUI (enlaces de comando y botones, botones de opción y casillas de verificación, barras de desplazamiento, cuadros de cierre, etc.) son las unidades léxicas que forman el vocabulario del diseño de diálogo. Si cambias la apariencia o el comportamiento de estas unidades, es como si de repente inyectases palabras extranjeras en una comunicación en lenguaje natural.
			
\clearpage
\thispagestyle{plain}
			
	\subsubsection{Inconsistencias}
		\par 
			Los controles de GUI no estándar son un caso especial del problema general de diseño inconsistente.
			
		\par \noindent
			La confusión se produce cuando las aplicaciones usan palabras o comandos diferentes para la misma cosa, o cuando usan la misma palabra para múltiples conceptos en diferentes partes de la aplicación. Del mismo modo, los usuarios se confunden cuando las cosas se mueven, violando la inercia de la pantalla. Usar el mismo nombre para la misma cosa en el mismo lugar facilita las cosas.
			
	\subsubsection{Mensajes de Error Incongruentes}
		\par 
			Los mensajes de error son una forma especial de comentarios: le dicen a los usuarios que algo salió mal. Conocemos las pautas para los mensajes de error desde hace casi 30 años y, sin embargo, muchas aplicaciones aún los infringen.
			
		\par \noindent
			La violación más común a las pautas es cuando un mensaje de error simplemente dice que algo está mal, sin explicar por qué y cómo el usuario puede solucionar el problema. Dichos mensajes dejan a los usuarios abandonados.
		
		\par \noindent
			Los mensajes de error informativos no solo ayudan a los usuarios a solucionar sus problemas actuales, sino que también pueden servir como un momento de aprendizaje. Por lo general, los usuarios no invertirán tiempo en leer y aprender sobre funciones, pero pasarán el tiempo para comprender una situación de error si lo explican claramente, porque quieren superar el error.
			
	\subsubsection{Preguntar por la misma información dos veces}
		\par
			Los usuarios no deberían tener que ingresar la misma información más de una vez. Después de todo, las computadoras son bastante buenas para recordar datos. La única razón por la que los usuarios tienen que repetir es porque los programadores se vuelven perezosos y no transfieren las respuestas de una parte de la aplicación a otra.
			
\clearpage
\thispagestyle{plain}

	\subsubsection{Sin valores por defecto}
		\par 
			Los valores predeterminados ayudan a los usuarios de muchas maneras. Lo más importante es que los valores predeterminados pueden:
		
		\begin{itemize}
			\item Acelerar la interacción al liberar a los usuarios de tener que especificar un valor si el valor predeterminado es aceptable.
			
			\item Enseñar, por ejemplo, el tipo de respuesta que es apropiada para la pregunta.
			
			\item Orientar a los usuarios novatos hacia un resultado seguro o común, permitiéndoles aceptar el valor predeterminado si no saben qué hacer.
		\end{itemize}
	