\thispagestyle{plain}
\subsection{Usabilidad}
	\par 
		La usabilidad es un atributo de calidad que evalúa qué tan fáciles de usar son las interfaces de usuario. La palabra "usabilidad" también se refiere a métodos para mejorar la facilidad de uso durante el proceso de diseño.
	
	\par \noindent
		Usabilidad es definido por 5 componentes de calidad:
		
	\begin{itemize}
		\item Capacidad de aprendizaje: ¿Qué tan fácil es para los usuarios realizar tareas básicas la primera vez que se encuentran con el diseño?
		
		\item Eficiencia: una vez que los usuarios han aprendido el diseño, ¿con qué rapidez pueden realizar tareas?
		
		\item Memorabilidad: cuando los usuarios regresan al diseño después de un período de no usarlo, ¿con qué facilidad pueden restablecer el dominio?
		
		\item Errores: ¿Cuántos errores cometen los usuarios, qué tan severos son estos errores y con qué facilidad pueden recuperarse de los errores?
		
		\item Satisfacción: ¿Qué tan agradable es usar el diseño?
	\end{itemize}

\clearpage
\thispagestyle{plain}

	\par \noindent
		Hay muchos otros atributos de calidad importantes. Una clave es la utilidad, que se refiere a la funcionalidad del diseño: ¿Hace lo que los usuarios necesitan?
		
	\par \noindent
		La usabilidad y la utilidad son igualmente importantes y juntos determinan si algo es útil: poco importa que algo sea fácil de usar si no hace lo que quieres. Tampoco es bueno si el sistema hipotéticamente puede hacer lo que quiere, pero no puede hacerlo porque la interfaz de usuario es demasiado difícil. Para estudiar la utilidad de un diseño, se pueden usar los mismos métodos de investigación de usuarios que mejoran la usabilidad.
		
	\begin{itemize}
		\item Utilidad: si proporciona las características que se necesitan.
		
		\item Usabilidad: qué tan fáciles y agradables son estas funciones para usar.
		
		\item Útil = Utilidad + Usabilidad.
	\end{itemize}

	\subsubsection{Como mejorar la usabilidad}
		\par 
			Existen muchos métodos para estudiar la usabilidad, pero el más básico y útil es la prueba del usuario, que tiene 3 componentes:
		\begin{itemize}
			\item Comunicación con algunos usuarios representativos.
			
			\item Solicitar a los usuarios que realicen tareas representativas con el diseño.
			
			\item Observar lo que hacen los usuarios, dónde tienen éxito y dónde tienen dificultades con la interfaz de usuario. Documentar lo que hacen los usuarios.
		\end{itemize}	
	
		\par \noindent
			Es importante probar a los usuarios individualmente y dejar que resuelvan cualquier problema por sí mismos. Si los ayuda o dirige su atención a alguna parte particular de la pantalla, los resultados de la prueba no serán validos.
			
		\par \noindent
			Para identificar los problemas de usabilidad más importantes de un diseño, probar 5 usuarios suele ser suficiente. En lugar de ejecutar un estudio grande y costoso, es mejor utilizar los recursos para realizar muchas pruebas pequeñas y revisar el diseño entre cada una para que pueda corregir los defectos de usabilidad a medida que los identifica. El diseño iterativo es la mejor manera de aumentar la calidad de la experiencia del usuario. Cuantas más versiones e ideas de interfaz pruebe con los usuarios, mejor.
			
\clearpage
\thispagestyle{plain}

	\subsubsection{Cuando trabajar en la usabilidad}
		\par 
			La usabilidad juega un papel en cada etapa del proceso de diseño. La necesidad resultante de múltiples estudios es una de las razones por las que Jakob Nielsen recomienda que los estudios individuales sean rápidos y económicos. Estos son los pasos principales:
			
		\begin{itemize}
			\item Antes de comenzar el nuevo diseño, pruebe el diseño anterior para identificar las partes buenas que debe conservar o destacar, y las partes defectuosas que causan problemas a los usuarios.
			
			\item Realice un estudio de campo para ver cómo se comportan los usuarios en su hábitat natural.
			
			\item Haga prototipos en papel de una o más ideas de diseño nuevas y pruébelas. Cuanto menos tiempo invierta en estas ideas de diseño, mejor, porque tendrá que cambiarlas todas según los resultados de la prueba.
			
			\item Refina las ideas de diseño que mejor se prueban a través de múltiples iteraciones, pasando gradualmente de prototipos de baja fidelidad a representaciones de alta fidelidad que se ejecutan en la computadora. Prueba cada iteración.
			
			\item Inspeccione el diseño en relación con las pautas de usabilidad establecidas, ya sea de sus propios estudios anteriores o investigación publicada.
			
			\item Una vez que decida e implemente el diseño final, pruébelo nuevamente. Los sutiles problemas de usabilidad siempre se infiltran durante la implementación.
		\end{itemize}
	
	\par \noindent
		No posponga las pruebas de usuario hasta que tenga un diseño completamente implementado. Si lo hace, será imposible solucionar la gran mayoría de los problemas críticos de usabilidad que descubre la prueba. Es probable que muchos de estos problemas sean estructurales, y corregirlos requeriría una mayor reconstrucción.
		
	\par \noindent
		La única forma de obtener una experiencia de usuario de alta calidad es iniciar las pruebas de los usuarios desde el principio del proceso de diseño y seguir realizando pruebas en cada paso del proceso.
		
\clearpage
\thispagestyle{plain}

	\subsubsection{Donde realizar pruebas}
		\par
			Para la mayoría de las compañías, sin embargo, está bien realizar pruebas en una sala de conferencias o en una oficina, siempre que pueda cerrar la puerta para evitar distracciones. Lo que importa es que consigas a los usuarios reales y te sientes con ellos mientras usan el diseño. Lapiz y papel es lo unico que se necesita.
	