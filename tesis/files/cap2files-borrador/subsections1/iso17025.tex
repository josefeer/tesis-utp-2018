\subsection{ISO 17025:2005}
	\subsubsection*{Requisitos generales para la competencia de los laboratorios de ensayo y de calibración}
	
	\subsubsection{Generalidades}
		\par 
			La primera edición (1999) de esta Norma Internacional fue producto de la amplia experiencia adquirida en la
			implementación de la Guía ISO/IEC 25 y de la Norma EN 45001, a las que reemplazó. Contiene todos los
			requisitos que tienen que cumplir los laboratorios de ensayo y de calibración si desean demostrar que poseen
			un sistema de gestión, son técnicamente competentes y son capaces de generar resultados técnicamente
			válidos. Es conveniente que los organismos de acreditación que reconocen la competencia de los laboratorios de
			ensayo y de calibración se basen en esta Norma Internacional para sus acreditaciones.
			
		\par \noindent
			El creciente uso de los sistemas de gestión ha producido un aumento de la necesidad de asegurar que los
			laboratorios que forman parte de organizaciones mayores o que ofrecen otros servicios, puedan funcionar de
			acuerdo con un sistema de gestión de la calidad que se considera que cumple la Norma ISO 9001 así como
			esta Norma Internacional. Por ello, se ha tenido el cuidado de incorporar todos aquellos requisitos de la
			Norma ISO 9001 que son pertinentes al alcance de los servicios de ensayo y de calibración cubiertos por el
			sistema de gestión del laboratorio.
		
		\par \noindent
			Los laboratorios de ensayo y de calibración que cumplen esta Norma Internacional funcionarán, por lo tanto,
			también de acuerdo con la Norma ISO 9001.
		
		\par \noindent
			La conformidad del sistema de gestión de la calidad implementado por el laboratorio, con los requisitos de la
			Norma ISO 9001, no constituye por sí sola una prueba de la competencia del laboratorio para producir datos
			y resultados técnicamente válidos. Por otro lado, la conformidad demostrada con esta Norma Internacional
			tampoco significa que el sistema de gestión de la calidad implementado por el laboratorio cumple todos los
			requisitos de la Norma ISO 9001.
			
\newpage
\thispagestyle{plain}
			
	\subsubsection{Objeto y campo de aplicación}
		\par
			Esta Norma Internacional establece los requisitos generales para la competencia en la realización de
			ensayos*) o de calibraciones, incluido el muestreo. Cubre los ensayos y las calibraciones que se realizan
			utilizando métodos normalizados, métodos no normalizados y métodos desarrollados por el propio laboratorio.
			
		\par \noindent
			Esta Norma Internacional es aplicable a todas las organizaciones que realizan ensayos o calibraciones.
			Éstas pueden ser, por ejemplo, los laboratorios de primera, segunda y tercera parte, y los laboratorios en los que los ensayos o las calibraciones forman parte de la inspección y la certificación de productos.
			
		\par \noindent
			Esta Norma Internacional es aplicable a todos los laboratorios, independientemente de la cantidad de
			empleados o de la extensión del alcance de las actividades de ensayo o de calibración. Cuando un laboratorio no realiza una o varias de las actividades contempladas en esta Norma Internacional, tales como el muestreo o el diseño y desarrollo de nuevos métodos, los requisitos de los apartados correspondientes no se aplican.
	
		\par \noindent
			Esta Norma Internacional es para que la utilicen los laboratorios cuando desarrollan los sistemas de
			gestión para sus actividades de la calidad, administrativas y técnicas. También puede ser utilizada por los clientes del laboratorio, las autoridades reglamentarias y los organismos de acreditación cuando confirman o reconocen la competencia de los laboratorios. Esta Norma Internacional no está destinada a ser utilizada
			como la base para la certificación de los laboratorios.
			
		\par \noindent
			El cumplimiento de los requisitos reglamentarios y de seguridad relacionados con el funcionamiento de
			los laboratorios no está cubierto por esta Norma Internacional.
			
		\par \noindent
			Si los laboratorios de ensayo y de calibración cumplen los requisitos de esta Norma Internacional,
			actuarán bajo un sistema de gestión de la calidad para sus actividades de ensayo y de calibración que
			también cumplirá los principios de la Norma ISO 9001.

\newpage
\thispagestyle{plain}

	\subsubsection{Requisitos relativos a la gestión}
		\paragraph{Control de Registros} 
			\par 
				El laboratorio debe establecer y mantener procedimientos para la identificación, la recopilación,
				la codificación, el acceso, el archivo, el almacenamiento, el mantenimiento y la disposición de los registros de
				la calidad y los registros técnicos. Los registros de la calidad deben incluir los informes de las auditorías
				internas y de las revisiones por la dirección, así como los registros de las acciones correctivas y preventivas.
				
			\par \noindent
				Todos los registros deben ser legibles y se deben almacenar y conservar de modo que sean
				fácilmente recuperables en instalaciones que les provean un ambiente adecuado para prevenir los daños, el deterioro y las pérdidas. Se debe establecer el tiempo de retención de los registros. Los registros se pueden presentar sobre cualquier tipo de soporte, tal como papel o soporte informático.
				
			\par \noindent
				Todos los registros deben ser conservados en sitio seguro y en confidencialidad.
				
			\par \noindent
				El laboratorio debe tener procedimientos para proteger y salvaguardar los registros almacenados
				electrónicamente y para prevenir el acceso no autorizado o la modificación de dichos registros.
			
			\subparagraph{Registros Técnicos}
				\par 
					El laboratorio debe conservar, por un período determinado, los registros de las observaciones
					originales, de los datos derivados y de información suficiente para establecer un protocolo de control, los
					registros de calibración, los registros del personal y una copia de cada informe de ensayos o certificado de
					calibración emitido. Los registros correspondientes a cada ensayo o calibración deben contener suficiente
					información para facilitar, cuando sea posible, la identificación de los factores que afectan a la incertidumbre y
					posibilitar que el ensayo o la calibración sea repetido bajo condiciones lo más cercanas posible a las
					originales. Los registros deben incluir la identidad del personal responsable del muestreo, de la realización de
					cada ensayo o calibración y de la verificación de los resultados.
					
					\par \noindent
						Las observaciones, los datos y los cálculos se deben registrar en el momento de hacerlos y
						deben poder ser relacionados con la operación en cuestión.
						
\newpage
\thispagestyle{plain}
						
					\par \noindent
						Cuando ocurran errores en los registros, cada error debe ser tachado, no debe ser borrado,
						hecho ilegible ni eliminado, y el valor correcto debe ser escrito al margen. Todas estas alteraciones a los
						registros deben ser firmadas o visadas por la persona que hace la corrección. En el caso de los registros
						guardados electrónicamente, se deben tomar medidas similares para evitar pérdida o cambio de los datos
						originales.
						
	\subsubsection{Requisitos técnicos}
		\par 
			Muchos factores determinan la exactitud y la confiabilidad de los ensayos o de las calibraciones
			realizados por un laboratorio. Estos factores incluyen elementos provenientes:
			
		\begin{itemize}
			\item Factores humanos
			
			\item Instalaciones y condiciones ambientales
			
			\item Métodos de ensayo y calibración, y de la validación de los métodos.
			
			\item Equipos
			
			\item Trazabilidad de las mediciones 
			
			\item Muestreo
			
			\item Manipulación de los item de ensayo y de la calibración.
		\end{itemize}

		\par \noindent
			El grado con el que los factores contribuyen a la incertidumbre total de la medición difiere
			considerablemente según los ensayos (y tipos de ensayos) y calibraciones (y tipos de calibraciones). El
			laboratorio debe tener en cuenta estos factores al desarrollar los métodos y procedimientos de ensayo y de
			calibración, en la formación y la calificación del personal, así como en la selección y la calibración de los
			equipos utilizados.
			
\newpage
\thispagestyle{plain}
		
		\paragraph{Personal (Factores Humanos)}
			\par 
				La dirección del laboratorio debe asegurar la competencia de todos los que operan equipos
				específicos, realizan ensayos o calibraciones, evalúan los resultados y firman los informes de ensayos y los
				certificados de calibración. Cuando emplea personal en formación, debe proveer una supervisión apropiada.
				El personal que realiza tareas específicas debe estar calificado sobre la base de una educación, una
				formación, una experiencia apropiadas y de habilidades demostradas, según sea requerido.
			
			\par \noindent
				La dirección del laboratorio debe formular las metas con respecto a la educación, la formación y las
				habilidades del personal del laboratorio. El laboratorio debe tener una política y procedimientos para
				identificar las necesidades de formación del personal y para proporcionarla. El programa de formación debe
				ser pertinente a las tareas presentes y futuras del laboratorio. Se debe evaluar la eficacia de las acciones de
				formación implementadas.
				
			\par \noindent
				El laboratorio debe disponer de personal que esté empleado por el laboratorio o que esté bajo
				contrato con él. Cuando utilice personal técnico y de apoyo clave, ya sea bajo contrato o a título
				suplementario, el laboratorio debe asegurarse de que dicho personal sea supervisado, que sea competente, y
				que trabaje de acuerdo con el sistema de gestión del laboratorio.
				
			\par \noindent
				El laboratorio debe mantener actualizados los perfiles de los puestos de trabajo del personal directivo,
				técnico y de apoyo clave involucrado en los ensayos o las calibraciones.
				
			\par \noindent
				La dirección debe autorizar a miembros específicos del personal para realizar tipos particulares
				de muestreos, ensayos o calibraciones, para emitir informes de ensayos y certificados de calibración, para
				emitir opiniones e interpretaciones y para operar tipos particulares de equipos. El laboratorio debe mantener
				registros de las autorizaciones pertinentes, de la competencia, del nivel de estudios y de las calificaciones
				profesionales, de la formación, de las habilidades y de la experiencia de todo el personal técnico, incluido el
				personal contratado. Esta información debe estar fácilmente disponible y debe incluir la fecha en la que se
				confirma la autorización o la competencia.
				
\newpage
\thispagestyle{plain}

		\paragraph{Instalaciones y condiciones ambientales}
			\par 
				Las instalaciones de ensayos o de calibraciones del laboratorio, incluidas, pero no de manera
				excluyente, las fuentes de energía, la iluminación y las condiciones ambientales, deben facilitar la realización
				correcta de los ensayos o de las calibraciones.
			
			\par \noindent
				El laboratorio debe asegurarse de que las condiciones ambientales no invaliden los resultados ni
				comprometan la calidad requerida de las mediciones. Se deben tomar precauciones especiales cuando el
				muestreo y los ensayos o las calibraciones se realicen en sitios distintos de la instalación permanente del
				laboratorio. Los requisitos técnicos para las instalaciones y las condiciones ambientales que puedan afectar a
				los resultados de los ensayos y de las calibraciones deben estar documentados.
				
			\par \noindent
				El laboratorio debe realizar el seguimiento, controlar y registrar las condiciones ambientales según lo
				requieran las especificaciones, métodos y procedimientos correspondientes, o cuando éstas puedan influir en
				la calidad de los resultados. Se debe prestar especial atención, por ejemplo, a la esterilidad biológica, el polvo,
				la interferencia electromagnética, la radiación, la humedad, el suministro eléctrico, la temperatura y a los
				niveles de ruido y vibración, en función de las actividades técnicas en cuestión. Cuando las condiciones
				ambientales comprometan los resultados de los ensayos o de las calibraciones, éstos se deben interrumpir.
				
		\paragraph{Métodos de ensayo y de calibración y validación de los métodos}
			\par 
				El laboratorio debe aplicar métodos y procedimientos apropiados para todos los ensayos o las calibraciones
				dentro de su alcance. Estos incluyen el muestreo, la manipulación, el transporte, el almacenamiento y la
				preparación de los ítems a ensayar o a calibrar y, cuando corresponda, la estimación de la incertidumbre de
				la medición así como técnicas estadísticas para el análisis de los datos de los ensayos o de las calibraciones.
				
			\par \noindent
				El laboratorio debe tener instrucciones para el uso y el funcionamiento de todo el equipamiento pertinente, y
				para la manipulación y la preparación de los ítems a ensayar o a calibrar, o ambos, cuando la ausencia de
				tales instrucciones pudieran comprometer los resultados de los ensayos o de las calibraciones. Todas las
				instrucciones, normas, manuales y datos de referencia correspondientes al trabajo del laboratorio se deben
				mantener actualizados y deben estar fácilmente disponibles para el personal (véase 4.3). Las desviaciones
				respecto de los métodos de ensayo y de calibración deben ocurrir solamente si la desviación ha sido
				documentada, justificada técnicamente, autorizada y aceptada por el cliente.

\newpage
\thispagestyle{plain}
			
			\subparagraph{Selección de métodos}
				\par 
					El laboratorio debe utilizar los métodos de ensayo o de calibración, incluidos los de muestreo, que satisfagan
					las necesidades del cliente y que sean apropiados para los ensayos o las calibraciones que realiza. Se deben
					utilizar preferentemente los métodos publicados como normas internacionales, regionales o nacionales. 
				
				\par \noindent
					Cuando el cliente no especifique el método a utilizar, el laboratorio debe seleccionar los métodos apropiados que
					hayan sido publicados en normas internacionales, regionales o nacionales, por organizaciones técnicas
					reconocidas, o en libros o revistas científicas especializados, o especificados por el fabricante del equipo. También
					se pueden utilizar los métodos desarrollados por el laboratorio o los métodos adoptados por el laboratorio si son
					apropiados para el uso previsto y si han sido validados. El cliente debe ser informado del método elegido. El
					laboratorio debe confirmar que puede aplicar correctamente los métodos normalizados antes de utilizarlos para los
					ensayos o las calibraciones.
					
				\par \noindent
					Si el método propuesto por el cliente se considera inapropiado o desactualizado, el laboratorio debe
					informárselo.
					
			\subparagraph{Métodos desarrollados por el laboratorio}
				\par 
					La introducción de los métodos de ensayo y de calibración desarrollados por el laboratorio para su propio uso
					debe ser una actividad planificada y debe ser asignada a personal calificado, provisto de los recursos adecuados
				
				\par \noindent
					Los planes deben ser actualizados a medida que avanza el desarrollo y se debe asegurar una comunicación
					eficaz entre todo el personal involucrado.
			
			\subparagraph{Validación de los métodos}
				\par 
					La validación es la confirmación, a través del examen y el aporte de evidencias objetivas, de que
					se cumplen los requisitos particulares para un uso específico previsto.
					
\newpage
\thispagestyle{plain}			
					
				\par \noindent
					El laboratorio debe validar los métodos no normalizados, los métodos que diseña o desarrolla,
					los métodos normalizados empleados fuera del alcance previsto, así como las ampliaciones y modificaciones
					de los métodos normalizados, para confirmar que los métodos son aptos para el fin previsto. La validación
					debe ser tan amplia como sea necesario para satisfacer las necesidades del tipo de aplicación o del campo
					de aplicación dados. El laboratorio debe registrar los resultados obtenidos, el procedimiento utilizado para la
					validación y una declaración sobre la aptitud del método para el uso previsto.
					
				\par \noindent
					Es conveniente utilizar una o varias de las técnicas siguientes para la determinación del desempeño de un
					método:
				
				\begin{itemize}
					\item calibración utilizando patrones de referencia o materiales de referencia;
					
					\item comparación con resultados obtenidos con otros métodos;
					
					\item comparaciones interlaboratorios;
					
					\item evaluación sistemática de los factores que influyen en el resultado;
					
					\item evaluación de la incertidumbre de los resultados basada en el conocimiento científico de los principios teóricos del
					método y en la experiencia práctica.
				\end{itemize} 
				
				\par \noindent
					La gama y la exactitud de los valores que se obtienen empleando métodos validados (por
					ejemplo, la incertidumbre de los resultados, el límite de detección, la selectividad del método, la linealidad, el
					límite de repetibilidad o de reproducibilidad, la robustez ante influencias externas o la sensibilidad cruzada
					frente a las interferencias provenientes de la matriz de la muestra o del objeto de ensayo) tal como fueron
					fijadas para el uso previsto, deben responder a las necesidades de los clientes.
					
			\subparagraph{Estimación de la incertidumbre de medición}
				\par 
					Un laboratorio de calibración, o un laboratorio de ensayo que realiza sus propias calibraciones,
					debe tener y debe aplicar un procedimiento para estimar la incertidumbre de la medición para todas las
					calibraciones y todos los tipos de calibraciones.
		
\newpage
\thispagestyle{plain}
					
				\par \noindent
					Los laboratorios de ensayo deben tener y deben aplicar procedimientos para estimar la
					incertidumbre de la medición. En algunos casos la naturaleza del método de ensayo puede excluir un cálculo
					riguroso, metrológicamente y estadísticamente válido, de la incertidumbre de medición. En estos casos el
					laboratorio debe, por lo menos, tratar de identificar todos los componentes de la incertidumbre y hacer una
					estimación razonable, y debe asegurarse de que la forma de informar el resultado no dé una impresión
					equivocada de la incertidumbre. Una estimación razonable se debe basar en un conocimiento del desempeño
					del método y en el alcance de la medición y debe hacer uso, por ejemplo, de la experiencia adquirida y de los
					datos de validación anteriores.
					
				\par \noindent
					Cuando se estima la incertidumbre de la medición, se deben tener en cuenta todos los
					componentes de la incertidumbre que sean de importancia en la situación dada, utilizando métodos
					apropiados de análisis.
					
				\par \noindent
					Los cálculos y la transferencia de los datos deben estar sujetos a verificaciones adecuadas
					llevadas a cabo de una manera sistemática
					
				\par \noindent
					Cuando se utilicen computadoras o equipos automatizados para captar, procesar, registrar,
					informar, almacenar o recuperar los datos de los ensayos o de las calibraciones, el laboratorio debe
					asegurarse de que:
					
				\begin{itemize}
					\item el software desarrollado por el usuario esté documentado con el detalle suficiente y haya sido
					convenientemente validado, de modo que se pueda asegurar que es adecuado para el uso;
					
					\item se establecen e implementan procedimientos para proteger los datos; tales procedimientos deben incluir,
					pero no limitarse a, la integridad y la confidencialidad de la entrada o recopilación de los datos, su
					almacenamiento, transmisión y procesamiento;
					
					\item se hace el mantenimiento de las computadoras y equipos automatizados con el fin de asegurar que
					funcionan adecuadamente y que se encuentran en las condiciones ambientales y de operación
					necesarias para preservar la integridad de los datos de ensayo o de calibración.
				\end{itemize}
			
\newpage
\thispagestyle{plain}
			
		\paragraph{Equipos}
			\par 
				El laboratorio debe estar provisto con todos los equipos para el muestreo, la medición y el ensayo,
				requeridos para la correcta ejecución de los ensayos o de las calibraciones (incluido el muestreo, la
				preparación de los ítems de ensayo o de calibración y el procesamiento y análisis de los datos de ensayo o
				de calibración). En aquellos casos en los que el laboratorio necesite utilizar equipos que estén fuera de su
				control permanente, debe asegurarse de que se cumplan los requisitos de esta Norma Internacional.
				
			\par \noindent
				Los equipos y su software utilizado para los ensayos, las calibraciones y el muestreo deben permitir
				lograr la exactitud requerida y deben cumplir con las especificaciones pertinentes para los ensayos o las
				calibraciones concernientes. Se deben establecer programas de calibración para las magnitudes o los valores
				esenciales de los instrumentos cuando dichas propiedades afecten significativamente a los resultados. Antes
				de poner en servicio un equipo (incluido el utilizado para el muestreo) se debe calibrar o verificar con el fin de
				asegurar que responde a las exigencias especificadas del laboratorio y cumple las especificaciones
				normalizadas pertinentes. El equipo debe ser verificado o calibrado antes de su uso 
				
			\par \noindent
				Los equipos deben ser operados por personal autorizado. Las instrucciones actualizadas sobre el
				uso y el mantenimiento de los equipos (incluido cualquier manual pertinente suministrado por el fabricante del
				equipo) deben estar disponibles para ser utilizadas por el personal del laboratorio.
				
			\par \noindent
				Cada equipo y su software utilizado para los ensayos y las calibraciones, que sea importante para el
				resultado, debe, en la medida de lo posible, estar unívocamente identificado.
				
			\par \noindent
				Se deben establecer registros de cada componente del equipamiento y su software que sea
				importante para la realización de los ensayos o las calibraciones. Los registros deben incluir por lo menos lo
				siguiente:
				
			\begin{itemize}
				\item la identificación del equipo y de su software;
				
				\item el nombre del fabricante, la identificación del modelo, el número de serie u otra identificación única;
				
				\item las verificaciones de la conformidad del equipo con la especificación;
				
				\item la ubicación actual, cuando corresponda;
				
				\item las instrucciones del fabricante, si están disponibles, o la referencia a su ubicación;
				
\newpage
\thispagestyle{plain}
				
				\item las fechas, los resultados y las copias de los informes y de los certificados de todas las calibraciones, los
				ajustes, los criterios de aceptación, y la fecha prevista de la próxima calibración;
				
				\item el plan de mantenimiento, cuando corresponda, y el mantenimiento llevado a cabo hasta la fecha;
				
				\item todo daño, mal funcionamiento, modificación o reparación del equipo.
			\end{itemize}
		
			\par \noindent
				El laboratorio debe tener procedimientos para la manipulación segura, el transporte, el
				almacenamiento, el uso y el mantenimiento planificado de los equipos de medición con el fin de asegurar el
				funcionamiento correcto y de prevenir la contaminación o el deterioro.
				
			\par \noindent
				Los equipos que hayan sido sometidos a una sobrecarga o a un uso inadecuado, que den resultados
				dudosos, o se haya demostrado que son defectuosos o que están fuera de los límites especificados, deben
				ser puestos fuera de servicio. Se deben aislar para evitar su uso o se deben rotular o marcar claramente
				como que están fuera de servicio hasta que hayan sido reparados y se haya demostrado por calibración o
				ensayo que funcionan correctamente. El laboratorio debe examinar el efecto del defecto o desvío de los
				límites especificados en los ensayos o en las calibraciones anteriores y debe aplicar el procedimiento de
				"control del trabajo no conforme”.
				
			\par \noindent
				Cuando sea posible, todos los equipos bajo el control del laboratorio que requieran una calibración,
				deben ser rotulados, codificados o identificados de alguna manera para indicar el estado de calibración,
				incluida la fecha en la que fueron calibrados por última vez y su fecha de vencimiento o el criterio para la
				próxima calibración.
				
			\par \noindent
				Cuando, por cualquier razón, el equipo quede fuera del control directo del laboratorio, éste debe
				asegurarse de que se verifican el funcionamiento y el estado de calibración del equipo y de que son
				satisfactorios, antes de que el equipo sea reintegrado al servicio.
				
			\par \noindent
				Cuando se necesiten verificaciones intermedias para mantener la confianza en el estado de
				calibración de los equipos, éstas se deben efectuar según un procedimiento definido.
				
			\par \noindent
				Cuando las calibraciones den lugar a un conjunto de factores de corrección, el laboratorio debe tener
				procedimientos para asegurarse de que las copias (por ejemplo, en el software), se actualizan correctamente
				
			\par \noindent
				Se deben proteger los equipos de ensayo y de calibración, tanto el hardware como el software,
				contra ajustes que pudieran invalidar los resultados de los ensayos o de las calibraciones.
				
\newpage
\thispagestyle{plain}

		\paragraph{Trazabilidad de las mediciones}
			\par 
				Todos los equipos utilizados para los ensayos o las calibraciones, incluidos los equipos para mediciones
				auxiliares (por ejemplo, de las condiciones ambientales) que tengan un efecto significativo en la exactitud o
				en la validez del resultado del ensayo, de la calibración o del muestreo, deben ser calibrados antes de ser
				puestos en servicio. El laboratorio debe establecer un programa y un procedimiento para la calibración de sus
				equipos.
				
			\subparagraph{Calibración}
				\par 
					Para los laboratorios de calibración, el programa de calibración de los equipos debe ser diseñado
					y operado de modo que se asegure que las calibraciones y las mediciones hechas por el laboratorio sean
					trazables al Sistema Internacional de Unidades (SI).
				\par \noindent
					Un laboratorio de calibración establece la trazabilidad de sus propios patrones de medición e instrumentos de
					medición al sistema SI por medio de una cadena ininterrumpida de calibraciones o de comparaciones que los
					vinculen a los pertinentes patrones primarios de las unidades de medida SI. La vinculación a las unidades SI se
					puede lograr por referencia a los patrones de medición nacionales. Los patrones de medición nacionales
					pueden ser patrones primarios, que son realizaciones primarias de las unidades SI o representaciones
					acordadas de las unidades SI, basadas en constantes físicas fundamentales, o pueden ser patrones
					secundarios, que son patrones calibrados por otro instituto nacional de metrología. Cuando se utilicen servicios
					de calibración externos, se debe asegurar la trazabilidad de la medición mediante el uso de servicios de
					calibración provistos por laboratorios que puedan demostrar su competencia y su capacidad de medición y
					trazabilidad. Los certificados de calibración emitidos por estos laboratorios deben contener los resultados de la
					medición, incluida la incertidumbre de la medición o una declaración sobre la conformidad con una
					especificación metrológica identificada.
					
				\par \noindent
					Existen ciertas calibraciones que actualmente no se pueden hacer estrictamente en unidades SI.
					En estos casos la calibración debe proporcionar confianza en las mediciones al establecer la trazabilidad a
					patrones de medición apropiados, tales como:
					
\newpage
\thispagestyle{plain}

				\begin{itemize}
					\item el uso de materiales de referencia certificados provistos por un proveedor competente con el fin de
					caracterizar física o químicamente un material de manera confiable;
					
					\item la utilización de métodos especificados o de normas consensuadas, claramente descritos y acordados
					por todas las partes concernientes.
				\end{itemize}
			
				\par \noindent
					Siempre que sea posible se requiere la participación en un programa adecuado de comparaciones
					interlaboratorios.
					
			\subparagraph{Ensayos}
				\par 
					Para los laboratorios de ensayo, los requisitos dados en calibración se aplican al equipo de medición
					y de ensayo con funciones de medición utilizado, a menos que se haya establecido que la incertidumbre
					introducida por la calibración contribuye muy poco a la incertidumbre total del resultado de ensayo. Cuando
					se dé esta situación, el laboratorio debe asegurarse de que el equipo utilizado puede proveer la incertidumbre
					de medición requerida.
					
				\par \noindent
					Cuando la trazabilidad de las mediciones a las unidades SI no sea posible o no sea pertinente,
					se deben exigir los mismos requisitos para la trazabilidad (por ejemplo, por medio de materiales de referencia
					certificados, métodos acordados o normas consensuadas) que para los laboratorios de calibración.
					
			\subparagraph{Patrones de referencia y materiales de referencia}
				\begin{itemize}
					\item Patrones de referencia: El laboratorio debe tener un programa y un procedimiento para la calibración de sus patrones de referencia.
					Los patrones de referencia deben ser calibrados por un organismo que pueda proveer la trazabilidad. Dichos patrones de referencia para la medición, conservados por el laboratorio, deben ser
					utilizados sólo para la calibración y para ningún otro propósito, a menos que se pueda demostrar que su
					desempeño como patrones de referencia no será invalidado. Los patrones de referencia deben ser calibrados
					antes y después de cualquier ajuste.
					
					\item Materiales de referencia: Cada vez que sea posible se debe establecer la trazabilidad de los materiales de referencia a las unidades de
					medida SI o a materiales de referencia certificados. Los materiales de referencia internos deben ser
					verificados en la medida que sea técnica y económicamente posible
					
\newpage
\thispagestyle{plain}
					
					\item Verificaciones intermedias: Se deben llevar a cabo las verificaciones que sean necesarias para mantener la confianza en el estado de
					calibración de los patrones de referencia, primarios, de transferencia o de trabajo y de los materiales de
					referencia de acuerdo con procedimientos y una programación definidos.
					
					\item Transporte y almacenamiento: El laboratorio debe tener procedimientos para la manipulación segura, el transporte, el almacenamiento y el
					uso de los patrones de referencia y materiales de referencia con el fin de prevenir su contaminación o
					deterioro y preservar su integridad.
				\end{itemize}
			
		\paragraph{Aseguramiento de la calidad de los resultados de ensayo y calibración}
			\par 
				El laboratorio debe tener procedimientos de control de la calidad para realizar el seguimiento de la
				validez de los ensayos y las calibraciones llevados a cabo. Los datos resultantes deben ser registrados en
				forma tal que se puedan detectar las tendencias y, cuando sea posible, se deben aplicar técnicas estadísticas
				para la revisión de los resultados. Dicho seguimiento debe ser planificado y revisado y puede incluir, entre
				otros, los elementos siguientes:
				
			\begin{itemize}
				\item el uso regular de materiales de referencia certificados o un control de la calidad interno utilizando
				materiales de referencia secundarios;
				
				\item la participación en comparaciones interlaboratorios o programas de ensayos de aptitud;
				
				\item la repetición de ensayos o calibraciones utilizando el mismo método o métodos diferentes;
				
				\item la repetición del ensayo o de la calibración de los objetos retenidos;
				
				\item la correlación de los resultados para diferentes características de un ítem.
			\end{itemize}
		
			\par \noindent
				Los datos de control de la calidad deben ser analizados y, si no satisfacen los criterios predefinidos,
				se deben tomar las acciones planificadas para corregir el problema y evitar consignar resultados incorrectos.
			
\newpage
\thispagestyle{plain}

		\paragraph{Informes de ensayos}
			\par 
				Además de los requisitos indicados en el apartado 5.10.2, los informes de ensayos deben incluir,
				en los casos en que sea necesario para la interpretación de los resultados de los ensayos, lo siguiente:
				
			\begin{itemize}
				\item las desviaciones, adiciones o exclusiones del método de ensayo e información sobre condiciones de
				ensayo específicas, tales como las condiciones ambientales;
				
				\item cuando corresponda, una declaración sobre el cumplimiento o no cumplimiento con los requisitos o las
				especificaciones;
				
				\item cuando sea aplicable, una declaración sobre la incertidumbre de medición estimada; la información
				sobre la incertidumbre es necesaria en los informes de ensayo cuando sea pertinente para la validez o
				aplicación de los resultados de los ensayos, cuando así lo requieran las instrucciones del cliente, o
				cuando la incertidumbre afecte al cumplimiento con los límites de una especificación;
				
				\item cuando sea apropiado y necesario, las opiniones e interpretaciones;
				
				\item la información adicional que pueda ser requerida por métodos específicos, clientes o grupos de clientes.
			\end{itemize}
				
				
		