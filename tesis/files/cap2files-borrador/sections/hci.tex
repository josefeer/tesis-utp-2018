\thispagestyle{plain}
\section{HCI}
	\par 
		La Interacción Humano-Computadora (HCI), es el estudio de la interacción entre el ser
		humano, las computadoras y las tareas que se desarrollan; principalmente se enfoca a
		conocer cómo la gente y las computadoras pueden interactuar para llevar a cabo tareas
		por medio de sistemas y software. 
	
	\par \noindent
		HCI es una materia que se basa en algunos aspectos relevantes de la teoría y métodos
		de muchas disciplinas, dentro de las cuales se incluye ciencias físicas y sociales,
		ingeniería y arte. Existen importantes contribuciones a la HCI que se han tomado de las
		ciencias de la computación, psicología, matemáticas, artes gráficas, sociología,
		inteligencia artificial, lingüística, filosofía, antropología y ergonomía. 
		
	\par \noindent
		HCI incluye partes fundamentales de la ergonomía debido a que se preocupa por
		entender cómo las computadoras y el ser humano pueden interactuar para desarrollar
		tareas existentes y nuevas. La ergonomía en HCI se enfoca a buscar los aspectos de
		diseño en los sistemas computacionales para que tengan un uso más efectivo y eficiente,
		así como el desarrollo de principios, guías, métodos y herramientas para mejorar el
		diseño y desarrollo de los sistemas interactivos computacionales. 

		
	\thispagestyle{plain}
\subsection{Usabilidad}
	\par 
		La usabilidad es un atributo de calidad que evalúa qué tan fáciles de usar son las interfaces de usuario. La palabra "usabilidad" también se refiere a métodos para mejorar la facilidad de uso durante el proceso de diseño.
	
	\par \noindent
		Usabilidad es definido por 5 componentes de calidad:
		
	\begin{itemize}
		\item Capacidad de aprendizaje: ¿Qué tan fácil es para los usuarios realizar tareas básicas la primera vez que se encuentran con el diseño?
		
		\item Eficiencia: una vez que los usuarios han aprendido el diseño, ¿con qué rapidez pueden realizar tareas?
		
		\item Memorabilidad: cuando los usuarios regresan al diseño después de un período de no usarlo, ¿con qué facilidad pueden restablecer el dominio?
		
		\item Errores: ¿Cuántos errores cometen los usuarios, qué tan severos son estos errores y con qué facilidad pueden recuperarse de los errores?
		
		\item Satisfacción: ¿Qué tan agradable es usar el diseño?
	\end{itemize}

\clearpage
\thispagestyle{plain}

	\par \noindent
		Hay muchos otros atributos de calidad importantes. Una clave es la utilidad, que se refiere a la funcionalidad del diseño: ¿Hace lo que los usuarios necesitan?
		
	\par \noindent
		La usabilidad y la utilidad son igualmente importantes y juntos determinan si algo es útil: poco importa que algo sea fácil de usar si no hace lo que quieres. Tampoco es bueno si el sistema hipotéticamente puede hacer lo que quiere, pero no puede hacerlo porque la interfaz de usuario es demasiado difícil. Para estudiar la utilidad de un diseño, se pueden usar los mismos métodos de investigación de usuarios que mejoran la usabilidad.
		
	\begin{itemize}
		\item Utilidad: si proporciona las características que se necesitan.
		
		\item Usabilidad: qué tan fáciles y agradables son estas funciones para usar.
		
		\item Útil = Utilidad + Usabilidad.
	\end{itemize}

	\subsubsection{Como mejorar la usabilidad}
		\par 
			Existen muchos métodos para estudiar la usabilidad, pero el más básico y útil es la prueba del usuario, que tiene 3 componentes:
		\begin{itemize}
			\item Comunicación con algunos usuarios representativos.
			
			\item Solicitar a los usuarios que realicen tareas representativas con el diseño.
			
			\item Observar lo que hacen los usuarios, dónde tienen éxito y dónde tienen dificultades con la interfaz de usuario. Documentar lo que hacen los usuarios.
		\end{itemize}	
	
		\par \noindent
			Es importante probar a los usuarios individualmente y dejar que resuelvan cualquier problema por sí mismos. Si los ayuda o dirige su atención a alguna parte particular de la pantalla, los resultados de la prueba no serán validos.
			
		\par \noindent
			Para identificar los problemas de usabilidad más importantes de un diseño, probar 5 usuarios suele ser suficiente. En lugar de ejecutar un estudio grande y costoso, es mejor utilizar los recursos para realizar muchas pruebas pequeñas y revisar el diseño entre cada una para que pueda corregir los defectos de usabilidad a medida que los identifica. El diseño iterativo es la mejor manera de aumentar la calidad de la experiencia del usuario. Cuantas más versiones e ideas de interfaz pruebe con los usuarios, mejor.
			
\clearpage
\thispagestyle{plain}

	\subsubsection{Cuando trabajar en la usabilidad}
		\par 
			La usabilidad juega un papel en cada etapa del proceso de diseño. La necesidad resultante de múltiples estudios es una de las razones por las que Jakob Nielsen recomienda que los estudios individuales sean rápidos y económicos. Estos son los pasos principales:
			
		\begin{itemize}
			\item Antes de comenzar el nuevo diseño, pruebe el diseño anterior para identificar las partes buenas que debe conservar o destacar, y las partes defectuosas que causan problemas a los usuarios.
			
			\item Realice un estudio de campo para ver cómo se comportan los usuarios en su hábitat natural.
			
			\item Haga prototipos en papel de una o más ideas de diseño nuevas y pruébelas. Cuanto menos tiempo invierta en estas ideas de diseño, mejor, porque tendrá que cambiarlas todas según los resultados de la prueba.
			
			\item Refina las ideas de diseño que mejor se prueban a través de múltiples iteraciones, pasando gradualmente de prototipos de baja fidelidad a representaciones de alta fidelidad que se ejecutan en la computadora. Prueba cada iteración.
			
			\item Inspeccione el diseño en relación con las pautas de usabilidad establecidas, ya sea de sus propios estudios anteriores o investigación publicada.
			
			\item Una vez que decida e implemente el diseño final, pruébelo nuevamente. Los sutiles problemas de usabilidad siempre se infiltran durante la implementación.
		\end{itemize}
	
	\par \noindent
		No posponga las pruebas de usuario hasta que tenga un diseño completamente implementado. Si lo hace, será imposible solucionar la gran mayoría de los problemas críticos de usabilidad que descubre la prueba. Es probable que muchos de estos problemas sean estructurales, y corregirlos requeriría una mayor reconstrucción.
		
	\par \noindent
		La única forma de obtener una experiencia de usuario de alta calidad es iniciar las pruebas de los usuarios desde el principio del proceso de diseño y seguir realizando pruebas en cada paso del proceso.
		
\clearpage
\thispagestyle{plain}

	\subsubsection{Donde realizar pruebas}
		\par
			Para la mayoría de las compañías, sin embargo, está bien realizar pruebas en una sala de conferencias o en una oficina, siempre que pueda cerrar la puerta para evitar distracciones. Lo que importa es que consigas a los usuarios reales y te sientes con ellos mientras usan el diseño. Lapiz y papel es lo unico que se necesita.
	
	
	\subsection{Interfaz de usuario}
	\par 
		La interfaz de usuario es el espacio donde se producen las interacciones entre seres humanos y máquinas. El objetivo de esta interacción es permitir el funcionamiento y control más efectivo de la máquina desde la interacción con el humano.
		
	\par \noindent
		Las interfaces básicas de usuario son aquellas que incluyen elementos como menús, ventanas, contenido gráfico, cursor, los beeps y algunos otros sonidos que la computadora hace, y en general, todos aquellos canales por los cuales se permite la comunicación entre el ser humano y la computadora.
		
	\par \noindent
		El objetivo del diseño de una interfaz es producir una interfaz que sea fácil de usar (explicarse por sí misma), eficiente y agradable para que al operar la máquina dé el resultado deseado.
		
	\par \noindent
		Ahora que sabemos la definición de lo que es el interfaz de usuario y su función, ahora veremos características de una mala intefaz.
		
	\subsubsection{Controles de interfaz no estandarizados} 
		\par 
			Los widgets básicos de la GUI (enlaces de comando y botones, botones de opción y casillas de verificación, barras de desplazamiento, cuadros de cierre, etc.) son las unidades léxicas que forman el vocabulario del diseño de diálogo. Si cambias la apariencia o el comportamiento de estas unidades, es como si de repente inyectases palabras extranjeras en una comunicación en lenguaje natural.
			
\clearpage
\thispagestyle{plain}
			
	\subsubsection{Inconsistencias}
		\par 
			Los controles de GUI no estándar son un caso especial del problema general de diseño inconsistente.
			
		\par \noindent
			La confusión se produce cuando las aplicaciones usan palabras o comandos diferentes para la misma cosa, o cuando usan la misma palabra para múltiples conceptos en diferentes partes de la aplicación. Del mismo modo, los usuarios se confunden cuando las cosas se mueven, violando la inercia de la pantalla. Usar el mismo nombre para la misma cosa en el mismo lugar facilita las cosas.
			
	\subsubsection{Mensajes de Error Incongruentes}
		\par 
			Los mensajes de error son una forma especial de comentarios: le dicen a los usuarios que algo salió mal. Conocemos las pautas para los mensajes de error desde hace casi 30 años y, sin embargo, muchas aplicaciones aún los infringen.
			
		\par \noindent
			La violación más común a las pautas es cuando un mensaje de error simplemente dice que algo está mal, sin explicar por qué y cómo el usuario puede solucionar el problema. Dichos mensajes dejan a los usuarios abandonados.
		
		\par \noindent
			Los mensajes de error informativos no solo ayudan a los usuarios a solucionar sus problemas actuales, sino que también pueden servir como un momento de aprendizaje. Por lo general, los usuarios no invertirán tiempo en leer y aprender sobre funciones, pero pasarán el tiempo para comprender una situación de error si lo explican claramente, porque quieren superar el error.
			
	\subsubsection{Preguntar por la misma información dos veces}
		\par
			Los usuarios no deberían tener que ingresar la misma información más de una vez. Después de todo, las computadoras son bastante buenas para recordar datos. La única razón por la que los usuarios tienen que repetir es porque los programadores se vuelven perezosos y no transfieren las respuestas de una parte de la aplicación a otra.
			
\clearpage
\thispagestyle{plain}

	\subsubsection{Sin valores por defecto}
		\par 
			Los valores predeterminados ayudan a los usuarios de muchas maneras. Lo más importante es que los valores predeterminados pueden:
		
		\begin{itemize}
			\item Acelerar la interacción al liberar a los usuarios de tener que especificar un valor si el valor predeterminado es aceptable.
			
			\item Enseñar, por ejemplo, el tipo de respuesta que es apropiada para la pregunta.
			
			\item Orientar a los usuarios novatos hacia un resultado seguro o común, permitiéndoles aceptar el valor predeterminado si no saben qué hacer.
		\end{itemize}
	