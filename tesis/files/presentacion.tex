\thispagestyle{empty}
\begin{center}
	
	UNIVERSIDAD TECNOLÓGICA DE PANAMÁ
	
	\vfill
	
	FACULTAD DE INGENIERÍA DE SISTEMAS COMPUTACIONALES
	
	\vfill
	
	DEPARTAMENTO DE COMPUTACIÓN Y SIMULACIÓN DE SISTEMAS
	
	\vfill
	
	DESARROLLO DE UN PROTOTIPO DE SISTEMA DE MEDICIÓN DE TEMPERATURA PARA LA EMPRESA SIGCSA UTILIZANDO TECNOLOGÍA ARDUINO
	
	\vfill
	
	JOSE FERNANDO MORALES DE LEÓN
	
	\vfill
	
	M.Sc. DORIS ELENA GUTIÉRREZ ROSALES
	
	\vfill
	
	2018
		
\end{center}
\clearpage

\thispagestyle{empty}

\begin{center}

	UNIVERSIDAD TECNOLÓGICA DE PANAMÁ
	
	\vfill
	
	FACULTAD DE INGENIERÍA DE SISTEMAS COMPUTACIONALES
	
	\vfill
	
	DESARROLLO DE UN PROTOTIPO DE SISTEMA DE MEDICIÓN DE TEMPERATURA PARA LA EMPRESA SIGCSA UTILIZANDO TECNOLOGÍA ARDUINO
	
	\vfill
	ASESORA\baselineskip=1\baselineskip
	
	M.Sc. DORIS ELENA GUTIÉRREZ ROSALES
	
	\vfill
	
	JOSE FERNANDO MORALES DE LEÓN
	
	\vfill
	
	TRABAJO DE GRADUACIÓN PARA OPTAR AL TÍTULO DE
	
	\vfill
	
	LICENCIADO EN INGENIERÍA DE SISTEMAS Y COMPUTACIÓN
	
	\vfill
	
	2018

\end{center}

\clearpage

\thispagestyle{plain}

\begin{flushright}
	\vspace*{\fill}
	
	\textit{Dedico esto a mi familia, especialmente a mi abuela Emma De León Cabrera, gracias por su apoyo incondicional.}
	
	\vspace*{\fill}
\end{flushright}

\clearpage

\thispagestyle{plain}
\begin{center}
	\huge
	\textbf{Agradecimientos}
\end{center}

\par \noindent
Agradezco a Dios y a mi madre.

\clearpage

\thispagestyle{plain}
\begin{center}
	\huge
	\textbf{Resumen}
\end{center}

\par \noindent
La presente tesis presenta el diseño y desarrollo de un prototipo para la captura de medidas de temperatura. El proyecto busca reemplazar la actividad de capturar de datos por una persona y dejarle el trabajo meramente al dispositivo y su aplicación. La propuesta fue desarrollada luego de realizar una investigación del proceso actual que ejecuta la empresa Soluciones Integrales en Gestión de Calidad, S.A (SIGCSA) para la caracterización de medios isotermos, los estándares implementados en dicho proceso y recomendaciones del personal de la empresa.\\ 
El resultado final de este trabajo permitirá al personal de SIGCSA realizar otros labores; mientras su teléfono capturará las mediciones de temperaturas de los equipos a caracterizar, disminuyendo el tiempo de servicio técnico manteniendo la calidad de servicio a sus clientes.

\par \noindent
\textit{\textbf{Palabras Claves:} prototipo, caracterización, medios isotermos; mediciones de temperatura}

\clearpage
