\section{Proceso Operativo SIGCSA-PO-25}

\par 
El objetivo del proceso es de proporcionar instrucciones para la caracterización y ajustes (tomando en cuenta las instrucciones
del fabricante) de medios isotermos con el propósito de confirmar el correcto funcionamiento de
la cámara de ensayo de temperatura\cite{po25}.

\subsection{Alcance}
\par 
Para ser utilizados en Laboratorios Clínicos, de Ensayos, de Investigación y Bancos de Sangre, que
tengan medios isotermos de controles análogos y digitales tales como\cite{po25}:

\begin{itemize}
	\item Refrigerador de 0°C a 6°C (la conservación de sangre y derivados).
	
	\item Refrigerador de 0°C a 8°C (conservación de reactivos).
	
	\item Refrigerador de Baja Temperatura de 0°C a -35°C (conservación de reactivos y derivados
	de la sangre).
	
	\item Refrigerador de Ultra Baja Temperatura 0°C a -80°C (criopreservación de cepas y tejido
	biológico).
	
	\item Incubadoras de -10°C a 75°C (conservación de organismos vivos en un entorno que
	resulte adecuado para su crecimiento y conservación de derivados de la sangre). **
	
	\item Baño de María de temperatura ambiente a 60°C (incubación, inactivación, aglutinación y
	descongelación de derivados de la sangre).
	
	\item Baño Seco de temperatura ambiente a 60°C (incubación).
	
	\item Estufa u Horno de Secado de temperatura ambiente a 100°C (procesos de secado y
	esterilizado de recipientes de vidrio y metal).
\end{itemize}

\begin{figure}[H]
	\centering
	\includegraphics[width=5cm, height=4cm]{isotermo1.png}
	\caption{Medio isotermo 1, capacidad de 50L a 2100L}
\end{figure}

\begin{figure}[H]
	\centering
	\includegraphics[width=5cm, height=4cm]{isotermo2.png}
	\caption{Medio isotermo 2, capacidad de 2L a 49L}
\end{figure}
	
\subsection{Análisis}
\par 
La caracterización es el conjunto de operaciones que determinan las diferentes
características metrológicas y especificaciones de operación de un
equipo, instrumento de medición, sistema de medición o medida
materializada. Por lo que este proceso se encarga de obtener las variables metrológicas de un equipo isotérmico, utilizando termometros en ciertos puntos del equipo. 

\par \noindent
El procedimiento se puede resumir en 4 pasos:

\begin{itemize}
	\item Preparación: Se determina las incertidumbres de los termómetros utilizados para caracterizar el medio isotermo y para validar que los termómetros se encuentran en buenas condiciones. Se documenta y se valida que las condiciones ambientales donde se encuentra el medio isotérmico son las adecuadas. **
	
	\item Instalación: Si todos los parámetros del paso anterior cumplen, entonces procedemos a colocar las sondas, si el medio isotermo tiene una capacidad mayor a 49L se colocan las sondas como en la figura 3.1, en caso contrario como en la figura 3.2.
	
	\item Captura de Datos: Se documenta la temperatura y humedad inicial, se captura la información que despliegan los termómetros por el tiempo estipulado en el procedimiento y una vez finalizado el tiempo nuevamente se documenta la temperatura y humedad.
	
	\item Cálculos metrologicos: Una vez capturada la información, se procede a realizar los cálculos que determinaran las características metrológicas del equipo isotérmico. ** 
\end{itemize}

\begin{figure}[H]
	\centering
	\includegraphics[width=\textwidth]{diagram1.png}
	\caption{Diagrama de flujo del proceso operativo SIGCSA-PO-25}
\end{figure}