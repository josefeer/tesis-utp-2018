\pagestyle{plain}
\centering\textbf{DatabaseHelper.java}

\begin{lstlisting}[language=java, caption={Clase DatabaseHelper, crea la base de datos de no existir y contiene las consultas e inserciones a la misma}, captionpos=b, basicstyle=\small]

package com.prittysoft.jat;

import android.content.ContentValues;
import android.content.Context;
import android.database.Cursor;
import android.database.sqlite.SQLiteDatabase;
import android.database.sqlite.SQLiteOpenHelper;
import android.util.Log;

import java.util.ArrayList;


public class DatabaseHelper extends SQLiteOpenHelper {

private static final String TAG = "DatabaseHelper";

/*DB tables*/
private static final String DB_NAME = "JAT_DB";
private static final String TABLE1_NAME = "mainmeasurements";
private static final String TABLE2_NAME = "isotermos1";
private static final String TABLE3_NAME = "isotermos2";
private static final String TABLE4_NAME = "calibracionpatron";

/*TABLE1 fields*/
private static final String TABLE1_COL1 = "id_main";
private static final String TABLE1_COL2 = "date";
private static final String TABLE1_COL3 = "start_time";
private static final String TABLE1_COL4 = "end_time";
private static final String TABLE1_COL5 = "tipo_ensayo";
private static final String TABLE1_COL6 = "tiempo_ensayo";
private static final String TABLE1_COL7 = "tiempo_estabilizacion";
private static final String TABLE1_COL8 = "tiempo_captura";
private static final String TABLE1_COL9 = "equipo_nombre";
private static final String TABLE1_COL10 = "equipo_modelo";
private static final String TABLE1_COL11 = "equipo_serial";
private static final String TABLE1_COL12 = "equipo_cliente";
private static final String TABLE1_COL13 = "status";

/*TABLE2 fields*/
private static final String TABLE2_COL1 = "id_isotermos2";
private static final String TABLE2_COL2 = "id_main";
private static final String TABLE2_COL3 = "timestamp";
private static final String TABLE2_COL4 = "S1";
private static final String TABLE2_COL5 = "S2";
private static final String TABLE2_COL6 = "S3";
private static final String TABLE2_COL7 = "S4";

/*TABLE3 fields*/
private static final String TABLE3_COL1 = "id_isotermos1";
private static final String TABLE3_COL2 = "id_main";
private static final String TABLE3_COL3 = "timestamp";
private static final String TABLE3_COL4 = "S1";
private static final String TABLE3_COL5 = "S2";
private static final String TABLE3_COL6 = "S3";
private static final String TABLE3_COL7 = "S4";
private static final String TABLE3_COL8 = "S5";
private static final String TABLE3_COL9 = "S6";
private static final String TABLE3_COL10 = "S7";
private static final String TABLE3_COL11 = "S8";
private static final String TABLE3_COL12 = "S9";

/*Table4 fields*/
private static final String TABLE4_COL1 = "id_sensorscalibration";
private static final String TABLE4_COL2 = "id_main";
private static final String TABLE4_COL3 = "timestamp";
private static final String TABLE4_COL4 = "S1";

/*CREATE TABLE queries*/
private static final String CREATETABLE1 = "
CREATE TABLE IF NOT EXISTS " + TABLE1_NAME + " (" + 
TABLE1_COL1 +" INTEGER PRIMARY KEY AUTOINCREMENT, " + 
TABLE1_COL2 + " TEXT, " + TABLE1_COL3 + " TEXT, " + 
TABLE1_COL4 + " TEXT, " + TABLE1_COL5 + " TEXT, " + 
TABLE1_COL6 + " TEXT, "+ TABLE1_COL7 + " TEXT, " + 
TABLE1_COL8 + " TEXT, "+ TABLE1_COL9 + " TEXT, "+ 
TABLE1_COL10 + " TEXT, " + TABLE1_COL11 + " TEXT, " + 
TABLE1_COL12 +" TEXT, " + TABLE1_COL13 + " TEXT)";

private static final String CREATETABLE2 = "
CREATE TABLE IF NOT EXISTS " + TABLE2_NAME + " (" + 
TABLE2_COL1 + " INTEGER PRIMARY KEY AUTOINCREMENT, " + 
TABLE2_COL2 + " INTEGER, " + TABLE2_COL3 + " TEXT, " + 
TABLE2_COL4 + " TEXT, " + TABLE2_COL5 + " TEXT, " + 
TABLE2_COL6 + " TEXT, " + TABLE2_COL7 + " TEXT, " +
"FOREIGN KEY ("+ TABLE2_COL2 + ") REFERENCES " + 
TABLE1_NAME + " (" + TABLE1_COL1 + "))";



private static final String CREATETABLE3 = "
CREATE TABLE IF NOT EXISTS " + TABLE3_NAME + " (" + 
TABLE3_COL1 + " INTEGER PRIMARY KEY AUTOINCREMENT, " + 
TABLE3_COL2 + " INTEGER, " + TABLE3_COL3 + " TEXT, " + 
TABLE3_COL4 + " TEXT, " + TABLE3_COL5 + " TEXT, " + 
TABLE3_COL6 + " TEXT, " + TABLE3_COL7 + " TEXT," + 
TABLE3_COL8 + " TEXT, " + TABLE3_COL9 + " TEXT, " + 
TABLE3_COL10 + " TEXT, " + TABLE3_COL11 + " TEXT, " + 
TABLE3_COL12 +", " + "FOREIGN KEY (" + TABLE3_COL2 + 
") REFERENCES " + TABLE1_NAME + " (" + TABLE1_COL1 + "))";

private static final String CREATETABLE4 = 
"CREATE TABLE IF NOT EXISTS " + TABLE4_NAME + " (" + 
TABLE4_COL1 + " INTEGER PRIMARY KEY AUTOINCREMENT, " + 
TABLE4_COL2 + " INTEGER, " + TABLE4_COL3 + " TEXT, " + 
TABLE4_COL4 + " TEXT, FOREIGN KEY (" + TABLE4_COL2 + 
") REFERENCES " + TABLE1_NAME + " (" + TABLE1_COL1 + "))";


/*DROP TABLE queries*/
private static final String DROPTABLE1 = 
"DROP TABLE IF EXISTS " + TABLE1_NAME;

private static final String DROPTABLE2 = 
"DROP TABLE IF EXISTS " + TABLE2_NAME;

private static final String DROPTABLE3 = 
"DROP TABLE IF EXISTS " + TABLE3_NAME;

private static final String DROPTABLE4 = 
"DROP TABLE IF EXISTS " + TABLE4_NAME;


public DatabaseHelper(Context context){
super(context, DB_NAME, null,1);
}

@Override
public void onCreate(SQLiteDatabase db) {
	db.execSQL(CREATETABLE1);
	db.execSQL(CREATETABLE2);
	db.execSQL(CREATETABLE3);
	db.execSQL(CREATETABLE4);
}

@Override
public void onUpgrade
(SQLiteDatabase db, 
int oldVersion, 
int newVersion) {

	db.execSQL(DROPTABLE1);
	db.execSQL(DROPTABLE2);
	db.execSQL(DROPTABLE3);
	db.execSQL(DROPTABLE4);
	onCreate(db);

}


// INSERT QUERIES
public boolean addDataTable1(String[] values){

	SQLiteDatabase db = this.getWritableDatabase();
	
	ContentValues contentValues = new ContentValues();
	contentValues.put(TABLE1_COL2, values[0]);
	contentValues.put(TABLE1_COL3, values[1]);
	contentValues.put(TABLE1_COL4, values[2]);
	contentValues.put(TABLE1_COL5, values[3]);
	contentValues.put(TABLE1_COL6, values[4]);
	contentValues.put(TABLE1_COL7, values[5]);
	contentValues.put(TABLE1_COL8, values[6]);
	contentValues.put(TABLE1_COL9, values[7]);
	contentValues.put(TABLE1_COL10, values[8]);
	contentValues.put(TABLE1_COL11, values[9]);
	contentValues.put(TABLE1_COL12, values[10]);
	
	long result = db.insert(TABLE1_NAME, null, contentValues);
	
	if (result == -1) {
		return false;
	} else
		return true;
}

public void addDataTable2(String[] values){

	SQLiteDatabase db = this.getWritableDatabase();
	ContentValues contentValues = new ContentValues();
	
	contentValues.put(TABLE2_COL2, values[0]);
	contentValues.put(TABLE2_COL3, values[1]);
	contentValues.put(TABLE2_COL4, values[2]);
	contentValues.put(TABLE2_COL5, values[3]);
	contentValues.put(TABLE2_COL6, values[4]);
	contentValues.put(TABLE2_COL7, values[5]);
	
	db.insert(TABLE2_NAME, null, contentValues);
	db.close();

}

public void addDataTable3(String[] values){

	SQLiteDatabase db = this.getWritableDatabase();
	ContentValues contentValues = new ContentValues();
	
	contentValues.put(TABLE3_COL2, values[0]);
	contentValues.put(TABLE3_COL3, values[1]);
	contentValues.put(TABLE3_COL4, values[2]);
	contentValues.put(TABLE3_COL5, values[3]);
	contentValues.put(TABLE3_COL6, values[4]);
	contentValues.put(TABLE3_COL7, values[5]);
	contentValues.put(TABLE3_COL8, values[6]);
	contentValues.put(TABLE3_COL9, values[7]);
	contentValues.put(TABLE3_COL10, values[8]);
	contentValues.put(TABLE3_COL11, values[9]);
	contentValues.put(TABLE3_COL12, values[10]);
	
	db.insert(TABLE3_NAME, null, contentValues);
	db.close();

}

public void addDataTable4(String[] values){

	SQLiteDatabase db = this.getWritableDatabase();
	ContentValues contentValues = new ContentValues();
	
	contentValues.put(TABLE3_COL2, values[0]);
	contentValues.put(TABLE3_COL3, values[1]);
	contentValues.put(TABLE3_COL4, values[2]);
	
	db.insert(TABLE4_NAME, null, contentValues);
	db.close();

}

//SELECT QUERIES
public Cursor getRecentsData(){

	SQLiteDatabase db = this.getReadableDatabase();
	String query = "SELECT * FROM " + TABLE1_NAME + 
	" ORDER BY " + TABLE1_COL1 + " DESC LIMIT 10";
	
	return db.rawQuery(query, null);

}

public Cursor getDataBetweenDates(
	String startdate, 
	String enddate){
	
	SQLiteDatabase db = this.getReadableDatabase();
	String query = "select * from " + TABLE1_NAME + 
	" where date between '" + startdate +"' and '" + 
	enddate +"' order by " + TABLE1_COL1 + " DESC";
	
	return db.rawQuery(query, null);

}

public Integer getAutoIncrementMeasurements(){
	SQLiteDatabase db = this.getReadableDatabase();
	String query = 
	"SELECT seq FROM sqlite_sequence
	 WHERE name=\"mainmeasurements\"";
	Cursor data = db.rawQuery(query, null);
	Integer AutoIncrement;
	if (data.moveToNext()){
		AutoIncrement = data.getInt(0);
	}
	else{
		AutoIncrement = 0;
	}
	data.close();
	Log.d(TAG, "AutoIncrement Value: " + 
	Integer.toString(AutoIncrement));
	
	return AutoIncrement;
}

public Cursor getSondaValues(
	String id,
	String table, 
	String SondaNumber){
	
	String Sonda = "S" + SondaNumber;
	
	SQLiteDatabase db = this.getReadableDatabase();
	String query = "SELECT timestamp, " + Sonda + 
	" FROM " + table + " WHERE id_main = " + id;
	
	return db.rawQuery(query, null);

}

// UPDATE QUERIES
public void updateTable1Complete(Integer seq, String endtime){

	SQLiteDatabase db = this.getWritableDatabase();
	
	ContentValues newValue = new ContentValues();
	newValue.put("end_time", endtime);
	newValue.put("status", "COMPLETADO");
	
	db.update(TABLE1_NAME, newValue, "id_main ="+seq, null);
	db.close();

}

public void updateTable1Canceled(Integer seq, String endtime){

	SQLiteDatabase db = this.getWritableDatabase();
	
	ContentValues newValue = new ContentValues();
	newValue.put("end_time", endtime);
	newValue.put("status", "CANCELADO");
	
	db.update(TABLE1_NAME, newValue, "id_main ="+seq, null);
	db.close();

}

// OTROS QUERIES
public ArrayList<String> getMainMeasurementDetails(String id){
	ArrayList<String> values = new ArrayList<>();
	SQLiteDatabase db = this.getReadableDatabase();
	String query = 
	"SELECT * FROM mainmeasurements WHERE id_main = " + id;
	
	Cursor data = db.rawQuery(query, null);
	
	//Cursor should return only one row
	if (data.moveToNext()){
		values.add(data.getString(0));
		values.add(data.getString(1));
		values.add(data.getString(2));
		values.add(data.getString(3));
		values.add(data.getString(4));
		values.add(data.getString(5));
		values.add(data.getString(6));
		values.add(data.getString(7));
		values.add(data.getString(8));
		values.add(data.getString(9));
		values.add(data.getString(10));
		values.add(data.getString(11));
		values.add(data.getString(12));
	}
	else {
		values = null;
	}
	
	data.close();
	db.close();
	
	return values;
}


}

\end{lstlisting}

\clearpage

\centering\textbf{RegisterTempService.java}

\begin{lstlisting}[language=java, caption={Clase RegisterTempService, Valida que el usuario no haya presionado el boton cancelar y realiza inserciones y actualizaciones a la base de datos.}, captionpos=b, basicstyle=\small]

package com.prittysoft.jat;

import android.app.IntentService;
import android.content.Intent;
import android.os.SystemClock;
import android.support.annotation.Nullable;
import android.support.v4.content.LocalBroadcastManager;
import android.util.Log;

import org.json.JSONException;
import org.json.JSONObject;

import java.text.SimpleDateFormat;
import java.util.Date;
import java.util.Locale;


public class RegisterTempService extends IntentService {

private static boolean ServiceStatus;
private static final String TAG = "RegisterTempService";
private static final Integer minute = 1000 * 60;
private static Intent localintent = new Intent("RegisterTemp");
SimpleDateFormat hour_format = new SimpleDateFormat(
"k:mm:ss", Locale.US);

// SimpleDateFormat timezone = new SimpleDateFormat("z", Locale.US);

// Class helper
DatabaseHelper mDatabaseHelper;

// Variables
String fecha, hora, tipo_ensayo, tiempo_ensayo, 
tiempo_estabilizacion, tiempo_captura,
equipo_nombre, equipo_modelo, equipo_serial, equipo_cliente;

String tipo_ensayo_posicion;

public RegisterTempService() {
super("RegisterTempService");
}

@Override
public void onDestroy() {
ServiceStatus = false;
Log.d(TAG, "Serviced Destroyed");
super.onDestroy();
}

@Override
protected void onHandleIntent(@Nullable Intent intent) {

ServiceStatus = true;
Log.d(TAG, "Service Started!");

if (intent != null) {

getValuesFromAddfragment(intent);
Data2DB();

}

Log.d(TAG, "Service Finished");

}

private void getValuesFromAddfragment(Intent intent) {

fecha = intent.getStringExtra("fecha");
hora = intent.getStringExtra("hora");
tipo_ensayo = intent.getStringExtra("tipo_ensayo");
tiempo_ensayo = intent.getStringExtra("tiempo_ensayo");
tiempo_estabilizacion = intent.getStringExtra(
"tiempo_estabilizacion");
tiempo_captura = intent.getStringExtra("tiempo_captura");
equipo_nombre = intent.getStringExtra("equipo_nombre");
equipo_modelo = intent.getStringExtra("equipo_modelo");
equipo_serial = intent.getStringExtra("equipo_serial");
equipo_cliente = intent.getStringExtra("equipo_cliente");
tipo_ensayo_posicion = intent.getStringExtra(
"tipo_ensayo_posicion");

}

private void Data2DB() {

mDatabaseHelper = new DatabaseHelper(getApplicationContext());

Integer t_estabilizacion = Integer.parseInt(
tiempo_estabilizacion);
Integer t_ensayo_posicion = Integer.parseInt(
tipo_ensayo_posicion);
Integer current_id;

//Database Insertion for measurements table
String[] Table1Values = {
fecha,
hora,
null,
tipo_ensayo,
tiempo_ensayo,
tiempo_estabilizacion,
tiempo_captura,
equipo_nombre,
equipo_modelo,
equipo_serial,
equipo_cliente,
"INCOMPLETO"
};


if (mDatabaseHelper.addDataTable1(Table1Values)) {

Log.d(TAG, "Succesfull Table 1 INSERT");

// Pausamos el proceso por el 
tiempo de estabilizacion si es diferente de 0
if (!tiempo_estabilizacion.equals("0")) {
Log.d(TAG, "Tiempo de estabilizacion es: " +
 tiempo_estabilizacion);
SystemClock.sleep(t_estabilizacion);
}

// Verificamos que no se ha cancelado el servicio
if (!ServiceStatus) {
stopSelf();
Log.d(TAG, "Service Interrupted");
return;
}

current_id = mDatabaseHelper.getAutoIncrementMeasurements();

try {

switch (t_ensayo_posicion) {
case 0:
InsertIsotermos1(current_id);
break;
case 1:
InsertIsotermos2(current_id);
break;
case 2:
InsertCalPatron(current_id);
break;
}

} catch (JSONException e) {

Log.d(TAG, "JSONException" + e.toString());

}

if (ServiceStatus) {

// Actualizamos el valor endtime 
que indica cuando se finalizo el proceso
mDatabaseHelper.updateTable1Complete(
current_id, 
hour_format.format(new Date()));

} else {

// Actualizamos el valor endtime 
que indica cuando se cancelo el proceso
mDatabaseHelper.updateTable1Canceled(current_id, 
hour_format.format(new Date()));

}

} else {

Log.d(TAG, "Unsuccesfull Table1 INSERT");

}


}

private void InsertIsotermos1(Integer queryID) 
throws JSONException {

String current_timestamp;

Integer t_ensayo = Integer.parseInt(tiempo_ensayo);
Integer t_captura = Integer.parseInt(tiempo_captura);
Integer progressbar_percentage, current_time;

JSONObject JSONValues;

String S1, S2, S3, S4, S5, S6, S7, S8, S9;

for (current_time = 1; 
current_time <= t_ensayo; 
current_time += t_captura) {

SystemClock.sleep(minute * t_captura);

// Verificamos que no se ha cancelado el servicio
if (!ServiceStatus) {
stopSelf();
Log.d(TAG, "Service Interrupted");
return;
}

current_timestamp = hour_format.format(new Date());

JSONValues = BTsocketHandler.getBTdata();
S1 = JSONValues.getString("S1");
S2 = JSONValues.getString("S2");
S3 = JSONValues.getString("S3");
S4 = JSONValues.getString("S4");
S5 = JSONValues.getString("S5");
S6 = JSONValues.getString("S6");
S7 = JSONValues.getString("S7");
S8 = JSONValues.getString("S8");
S9 = JSONValues.getString("S9");

String[] table3_values = {
queryID.toString(),
current_timestamp,
S1,
S2,
S3,
S4,
S5,
S6,
S7,
S8,
S9
};

mDatabaseHelper.addDataTable3(table3_values);

// Verificamos que no se ha cancelado el servicio
if (!ServiceStatus) {

stopSelf();
Log.d(TAG, "Service Interrupted");
return;

} else {

progressbar_percentage = (current_time * 100) / t_ensayo;

// Enviamos informacion del proceso 
al progressbar del Addfragment
localintent.putExtra("progressbar_percentage",
progressbar_percentage.toString());
LocalBroadcastManager.
getInstance(getApplicationContext()).
sendBroadcast(localintent);

}

}


}

private void InsertIsotermos2(Integer queryID) throws JSONException {

String current_timestamp;

Integer t_ensayo = Integer.parseInt(tiempo_ensayo);
Integer t_captura = Integer.parseInt(tiempo_captura);
Integer progressbar_percentage, current_time;

JSONObject JSONValues;

String S1, S2, S3, S4;

for (current_time = 1; 
current_time <= t_ensayo; 
current_time += t_captura) {

SystemClock.sleep(minute * t_captura);

// Verificamos que no se ha cancelado el servicio
if (!ServiceStatus) {
stopSelf();
Log.d(TAG, "Service Interrupted");
return;
}

current_timestamp = hour_format.format(new Date());

JSONValues = BTsocketHandler.getBTdata();
S1 = JSONValues.getString("S1");
S2 = JSONValues.getString("S2");
S3 = JSONValues.getString("S3");
S4 = JSONValues.getString("S4");

String[] table2_values = {
queryID.toString(),
current_timestamp,
S1,
S2,
S3,
S4
};

mDatabaseHelper.addDataTable2(table2_values);

// Verificamos que no se ha cancelado el servicio
if (!ServiceStatus) {

stopSelf();
Log.d(TAG, "Service Interrupted");
return;

} else {

progressbar_percentage = (current_time * 100) / t_ensayo;

// Enviamos informacion del proceso 
al progressbar del Addfragment
localintent.putExtra("progressbar_percentage",
progressbar_percentage.toString());
LocalBroadcastManager.
getInstance(getApplicationContext()).
sendBroadcast(localintent);

}

}


}

private void InsertCalPatron(Integer queryID) 
throws JSONException {

String current_timestamp;

Integer t_ensayo = Integer.parseInt(tiempo_ensayo);
Integer t_captura = Integer.parseInt(tiempo_captura);
Integer progressbar_percentage, current_time;

JSONObject JSONValues;

String S1;

for (current_time = 1; 
current_time <= t_ensayo; 
current_time += t_captura) {

SystemClock.sleep(minute * t_captura);

// Verificamos que no se ha cancelado el servicio
if (!ServiceStatus) {
stopSelf();
Log.d(TAG, "Service Interrupted");
return;
}

current_timestamp = hour_format.format(new Date());

JSONValues = BTsocketHandler.getBTdata();
S1 = JSONValues.getString("S1");

String[] table4_values = {
queryID.toString(),
current_timestamp,
S1,
};

mDatabaseHelper.addDataTable4(table4_values);

// Verificamos que no se ha cancelado el servicio
if (!ServiceStatus) {

stopSelf();
Log.d(TAG, "Service Interrupted");
return;

} else {

progressbar_percentage = (current_time * 100) / t_ensayo;

// Enviamos informacion del 
proceso al progressbar del Addfragment
localintent.putExtra("progressbar_percentage",
progressbar_percentage.toString());
LocalBroadcastManager.
getInstance(getApplicationContext()).
sendBroadcast(localintent);

}

}

}

}

\end{lstlisting}