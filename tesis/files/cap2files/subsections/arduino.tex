\subsection{Arduino}

\par
Arduino es una plataforma de electrónica de código abierto basada en hardware y software fácil de usar. Las placas Arduino pueden leer entradas (luz en un sensor, un dedo en un botón o un mensaje de Twitter) y convertirlo en una salida, activar un motor, encender un LED y publicar algo en línea. Puede decirle a su placa qué hacer enviando un conjunto de instrucciones al microcontrolador en la placa. Para hacerlo, utiliza el lenguaje de programación Arduino (basado en \textquotedblleft Wiring\textquotedblright) y el software Arduino (IDE), basado en \textquotedblleft Processing \textquotedblright\cite{arduino-intro}.

\begin{figure}[H]
	\centering
	\includegraphics[width=5cm, height=4cm]{arduino1.png}
	\caption{Logo Oficial de Arduino}
\end{figure}

\par \noindent
Con los años, Arduino ha sido el cerebro de miles de proyectos, desde objetos cotidianos hasta complejos instrumentos científicos. Una comunidad mundial de fabricantes (estudiantes, aficionados, artistas, programadores y profesionales) se ha reunido en torno a esta plataforma de código abierto, sus contribuciones se han añadido a una increíble cantidad de conocimiento accesible que puede ser de gran ayuda para principiantes y expertos por igual\cite{arduino-intro}.