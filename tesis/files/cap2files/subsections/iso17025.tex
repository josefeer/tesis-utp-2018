\subsection{ISO 17025:2005 \cite{iso17025}}
\subsection*{Requisitos generales para la competencia de los laboratorios de ensayo y de calibración}

\par 
La primera edición (1999) de esta Norma Internacional fue producto de la amplia experiencia adquirida en la
implementación de la Guía ISO/IEC 25 y de la Norma EN 45001, a las que reemplazó. Contiene todos los
requisitos que tienen que cumplir los laboratorios de ensayo y de calibración se desean demostrar que poseen
un sistema de gestión, técnicamente competentes y capaces de generar resultados técnicamente
válidos. Es conveniente que los organismos de acreditación que reconocen la competencia de los laboratorios de
ensayo y de calibración se basen en esta Norma Internacional para sus acreditaciones.

\subsubsection{Objeto y campo de aplicación}
\par
Esta Norma Internacional establece los requisitos generales para la competencia en la realización de
ensayos o de calibraciones, incluido el muestreo. Cubre los ensayos y las calibraciones que se realizan
utilizando métodos normalizados, métodos no normalizados y métodos desarrollados por el propio laboratorio.

\par \noindent
Esta Norma Internacional es aplicable a todas las organizaciones que realizan ensayos o calibraciones.
Éstas pueden ser, por ejemplo, los laboratorios de primera, segunda y tercera parte, y los laboratorios en los que los ensayos o las calibraciones forman parte de la inspección y la certificación de productos.

\par \noindent
Esta Norma Internacional es aplicable a todos los laboratorios, independientemente de la cantidad de
empleados o de la extensión del alcance de las actividades de ensayo o de calibración. Cuando un laboratorio no realiza una o varias de las actividades contempladas en esta Norma Internacional, tales como el muestreo o el diseño y desarrollo de nuevos métodos, los requisitos de los apartados correspondientes no se aplican.

\par \noindent
Esta Norma Internacional es para que la utilicen los laboratorios cuando desarrollan los sistemas de
gestión para sus actividades de la calidad, administrativas y técnicas. También puede ser utilizada por los clientes del laboratorio, las autoridades reglamentarias y los organismos de acreditación cuando confirman o reconocen la competencia de los laboratorios. Esta Norma Internacional no está destinada a ser utilizada
como la base para la certificación de los laboratorios.

\par \noindent
El cumplimiento de los requisitos reglamentarios y de seguridad relacionados con el funcionamiento de
los laboratorios no está cubierto por esta Norma Internacional.

\subsubsection{Requisitos relativos a la gestión}

\begin{itemize}
	\item Control de Registros: El laboratorio debe establecer y mantener procedimientos para la identificación, la recopilación,
	la codificación, el acceso, el archivo, el almacenamiento, el mantenimiento y la disposición de los registros de
	la calidad y los registros técnicos. Los registros de la calidad deben incluir los informes de las auditorías
	internas y de las revisiones por la dirección, así como los registros de las acciones correctivas y preventivas.
	
	\par \noindent
	Todos los registros deben ser legibles y se deben almacenar y conservar de modo que sean
	fácilmente recuperables en instalaciones que les provean un ambiente adecuado para prevenir los daños, el deterioro y las pérdidas. Se debe establecer el tiempo de retención de los registros. Los registros se pueden presentar sobre cualquier tipo de soporte, tal como papel o soporte informático.
	
	\par \noindent
	Todos los registros deben ser conservados en sitio seguro y en confidencialidad.
	
	\par \noindent
	El laboratorio debe tener procedimientos para proteger y salvaguardar los registros almacenados
	electrónicamente y para prevenir el acceso no autorizado o la modificación de dichos registros.
	
	\item Registros Técnicos: El laboratorio debe conservar, por un período determinado, los registros de las observaciones
	originales, de los datos derivados y de información suficiente para establecer un protocolo de control, los
	registros de calibración, los registros del personal y una copia de cada informe de ensayos o certificado de
	calibración emitido. Los registros correspondientes a cada ensayo o calibración deben contener suficiente
	información para facilitar, cuando sea posible, la identificación de los factores que afectan a la incertidumbre y
	posibilitar que el ensayo o la calibración sea repetido bajo condiciones lo más cercanas posible a las
	originales. Los registros deben incluir la identidad del personal responsable del muestreo, de la realización de
	cada ensayo o calibración y de la verificación de los resultados.
	
	\par \noindent
	Las observaciones, los datos y los cálculos se deben registrar en el momento de hacerlos y
	deben poder ser relacionados con la operación en cuestión.
	
	\par \noindent
	Cuando ocurran errores en los registros, cada error debe ser tachado, no debe ser borrado,
	hecho ilegible ni eliminado, y el valor correcto debe ser escrito al margen. Todas estas alteraciones a los
	registros deben ser firmadas o visadas por la persona que hace la corrección. En el caso de los registros
	guardados electrónicamente, se deben tomar medidas similares para evitar pérdida o cambio de los datos
	originales.
\end{itemize}

\subsubsection{Requisitos técnicos}
Muchos factores determinan la exactitud y la confiabilidad de los ensayos o de las calibraciones
realizados por un laboratorio. Estos factores incluyen elementos provenientes:

\begin{itemize}
	\item Factores humanos: La dirección del laboratorio debe asegurar la competencia de todos los que operan equipos
	específicos, realizan ensayos o calibraciones, evalúan los resultados y firman los informes de ensayos y los
	certificados de calibración. Cuando emplea personal en formación, debe proveer una supervisión apropiada.
	El personal que realiza tareas específicas debe estar calificado sobre la base de una educación, una
	formación, una experiencia apropiadas y de habilidades demostradas, según sea requerido.
	
	\item Instalaciones y condiciones ambientales: Las instalaciones de ensayos o de calibraciones del laboratorio, incluidas, pero no de manera
	excluyente, las fuentes de energía, la iluminación y las condiciones ambientales, deben facilitar la realización
	correcta de los ensayos o de las calibraciones.
	
	\par \noindent
	El laboratorio debe asegurarse de que las condiciones ambientales no invaliden los resultados ni
	comprometan la calidad requerida de las mediciones. Se deben tomar precauciones especiales cuando el
	muestreo y los ensayos o las calibraciones se realicen en sitios distintos de la instalación permanente del
	laboratorio. Los requisitos técnicos para las instalaciones y las condiciones ambientales que puedan afectar a
	los resultados de los ensayos y de las calibraciones deben estar documentados.
	
	\par \noindent
	El laboratorio debe realizar el seguimiento, controlar y registrar las condiciones ambientales según lo
	requieran las especificaciones, métodos y procedimientos correspondientes, o cuando éstas puedan influir en
	la calidad de los resultados. Se debe prestar especial atención, por ejemplo, a la esterilidad biológica, el polvo,
	la interferencia electromagnética, la radiación, la humedad, el suministro eléctrico, la temperatura y a los
	niveles de ruido y vibración, en función de las actividades técnicas en cuestión. Cuando las condiciones
	ambientales comprometan los resultados de los ensayos o de las calibraciones, éstos se deben interrumpir.
	
	\item Métodos de ensayo y calibración, y de la validación de los métodos: El laboratorio debe aplicar métodos y procedimientos apropiados para todos los ensayos o las calibraciones dentro de su alcance. Estos incluyen el muestreo, la manipulación, el transporte, el almacenamiento y la preparación de los ítems a ensayar o a calibrar y, cuando corresponda, la estimación de la incertidumbre de la medición así como técnicas estadísticas para el análisis de los datos de los ensayos o de las calibraciones.
	
	\par \noindent
	El laboratorio debe tener instrucciones para el uso y el funcionamiento de todo el equipamiento pertinente, y
	para la manipulación y la preparación de los ítems a ensayar o a calibrar, o ambos, cuando la ausencia de
	tales instrucciones pudieran comprometer los resultados de los ensayos o de las calibraciones. Todas las
	instrucciones, normas, manuales y datos de referencia correspondientes al trabajo del laboratorio se deben
	mantener actualizados y deben estar fácilmente disponibles para el personal. 
	
	\item Equipos: El laboratorio debe estar provisto con todos los equipos para el muestreo, la medición y el ensayo,
	requeridos para la correcta ejecución de los ensayos o de las calibraciones (incluido el muestreo, la
	preparación de los ítems de ensayo o de calibración y el procesamiento y análisis de los datos de ensayo o
	de calibración). En aquellos casos en los que el laboratorio necesite utilizar equipos que estén fuera de su
	control permanente, debe asegurarse de que se cumplan los requisitos de esta Norma Internacional.
	
	\par \noindent
	Los equipos y su software utilizado para los ensayos, las calibraciones y el muestreo deben permitir
	lograr la exactitud requerida y deben cumplir con las especificaciones pertinentes para los ensayos o las
	calibraciones concernientes. Se deben establecer programas de calibración para las magnitudes o los valores
	esenciales de los instrumentos cuando dichas propiedades afecten significativamente a los resultados. Antes
	de poner en servicio un equipo (incluido el utilizado para el muestreo) se debe calibrar o verificar con el fin de
	asegurar que responde a las exigencias especificadas del laboratorio y cumple las especificaciones
	normalizadas pertinentes. El equipo debe ser verificado o calibrado antes de su uso.
	
	\item Trazabilidad de las mediciones: Todos los equipos utilizados para los ensayos o las calibraciones, incluidos los equipos para mediciones
	auxiliares (por ejemplo, de las condiciones ambientales) que tengan un efecto significativo en la exactitud o
	en la validez del resultado del ensayo, de la calibración o del muestreo, deben ser calibrados antes de ser
	puestos en servicio. El laboratorio debe establecer un programa y un procedimiento para la calibración de sus
	equipos.
	
\end{itemize}

\par \noindent
El grado con el que los factores contribuyen a la incertidumbre total de la medición difiere
considerablemente según los ensayos (y tipos de ensayos) y calibraciones (y tipos de calibraciones). El
laboratorio debe tener en cuenta estos factores al desarrollar los métodos y procedimientos de ensayo y de
calibración, en la formación y la calificación del personal, así como en la selección y la calibración de los
equipos utilizados.

\subsubsection{Aseguramiento de la calidad de los resultados de ensayo y calibración}

\par 
El laboratorio debe tener procedimientos de control de la calidad para realizar el seguimiento de la
validez de los ensayos y las calibraciones llevados a cabo. Los datos resultantes deben ser registrados en
forma tal que se puedan detectar las tendencias y, cuando sea posible, se deben aplicar técnicas estadísticas
para la revisión de los resultados. Dicho seguimiento debe ser planificado y revisado y puede incluir, entre
otros, los elementos siguientes:

\begin{itemize}
\item el uso regular de materiales de referencia certificados o un control de la calidad interno utilizando
materiales de referencia secundarios;

\item la participación en comparaciones interlaboratorios o programas de ensayos de aptitud;

\item la repetición de ensayos o calibraciones utilizando el mismo método o métodos diferentes;

\item la repetición del ensayo o de la calibración de los objetos retenidos;

\item la correlación de los resultados para diferentes características de un ítem.
\end{itemize}

\par \noindent
Los datos de control de la calidad deben ser analizados y, si no satisfacen los criterios predefinidos,
se deben tomar las acciones planificadas para corregir el problema y evitar consignar resultados incorrectos.

\subsubsection{Informes de ensayos}
\par 
Los informes de ensayos deben incluir, en los casos en que sea necesario para la interpretación de los resultados de los ensayos, lo siguiente:

\begin{itemize}
	\item las desviaciones, adiciones o exclusiones del método de ensayo e información sobre condiciones de
	ensayo específicas, tales como las condiciones ambientales;
	
	\item cuando corresponda, una declaración sobre el cumplimiento o no cumplimiento con los requisitos o las
	especificaciones;
	
	\item cuando sea aplicable, una declaración sobre la incertidumbre de medición estimada; la información
	sobre la incertidumbre es necesaria en los informes de ensayo cuando sea pertinente para la validez o
	aplicación de los resultados de los ensayos, cuando así lo requieran las instrucciones del cliente, o
	cuando la incertidumbre afecte al cumplimiento con los límites de una especificación;
	
	\item cuando sea apropiado y necesario, las opiniones e interpretaciones;
	
	\item la información adicional que pueda ser requerida por métodos específicos, clientes o grupos de clientes.
\end{itemize}
