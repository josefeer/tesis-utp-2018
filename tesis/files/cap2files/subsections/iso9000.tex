\subsection{ISO 9000:2015}
\subsection*{Sistemas de gestión de calidad: Fundamentos y Vocabulario}

\par 
Los conceptos y los principios de la gestión de la calidad descritos en esta Norma Internacional proporcionan a la organización la capacidad de cumplir los retos presentados por un entorno que es profundamente diferente al de décadas recientes. El contexto en el que trabaja una organización actualmente se caracteriza por el cambio acelerado, la globalización de los mercados, los recursos limitados y la aparición del conocimiento como un recurso principal. El impacto de la calidad se extiende más allá de la satisfacción del cliente puede tener además un impacto directo en la reputación de la organización.\cite{iso9000}

\par \noindent 
Todos los conceptos, principios y sus interrelaciones deberían verse como un conjunto y no aislados unos de otros. Un concepto o principio individual no es más importante que otro. En cada momento es crítico encontrar un balance correcto en su aplicación. Para leer en detalle esta norma ver el \textbf{Anexo 1}.

\par \noindent
A pesar de que la norma ISO 9000:2015 define los conceptos de sistemas de gestión de calidad también define los de calibración y medición. Debemos tener un conocimiento general de la norma ISO 17025:2005; ya que, si nuestro prototipo sigue los lineamientos de esta norma. Le da más credibilidad para su futura implementación en pruebas de campo con SIGCSA. La norma ISO 17025:2005, hoy en día, se encuentra en estado de retiro debido a la publicación de la nueva versión 2017. No obstante, sigue siendo una buena referencia la versión 2005 para el proyecto.\cite{iso9000}
