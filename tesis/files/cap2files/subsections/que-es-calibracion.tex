\subsection{¿Que es Calibración?}

\par 
La palabra "calibración" tiene diferentes significados dependiendo de la industria o el entorno en el que se utiliza. En la industria de prueba y medición, la calibración tiene un significado específico, que, en un nivel básico, es el acto de comparar un dispositivo bajo prueba (DUT por sus siglas en ingles) de un valor desconocido con un estándar de referencia de un valor conocido. Una persona típicamente realiza una calibración para determinar el error o verificar la exactitud del valor desconocido del DUT. Como ejemplo básico, puede realizar una calibración midiendo la temperatura de un termómetro DUT en agua en el punto de ebullición conocido (100 grados Celsius) para conocer el error del termómetro. Debido a que la determinación visual del momento exacto en que se alcanza el punto de ebullición puede ser impreciso, usted puede lograr un resultado más preciso colocando un termómetro de referencia calibrado, de un valor conocido preciso, en el agua para verificar el termómetro DUT\cite{calibracion-fluke}.

\par \noindent
Un siguiente paso lógico que puede ocurrir en un proceso de calibración puede ser ajustar o realzar el instrumento para reducir el error de medición. Técnicamente, el ajuste es un paso separado de la calibración\cite{calibracion-fluke}.

\subsubsection{Acreditación de Calibración \cite{calibracion-fluke}}

\par 
Cuando se realizan calibraciones, es importante poder confiar en el proceso por el cual se realizan. La acreditación de calibración brinda esa confianza. La acreditación le da confianza al propietario del instrumento de que la calibración se ha realizado correctamente.

\par \noindent
La acreditación de calibración significa que se ha revisado un proceso de calibración y se ha encontrado que cumple con los requisitos de metrología técnica y de calidad aceptados internacionalmente. La ISO / IEC 17025 es la norma de calidad de metrología internacional a la cual los laboratorios de calibración están acreditados.

\par \noindent
Los acuerdos internacionales garantizan que una vez que se acredita un proceso de calibración en un país, las calibraciones provenientes de ese proceso pueden aceptarse en todo el mundo sin ningún requisito de aceptación técnica adicional.

\subsubsection{Disciplinas de la Calibración \cite{calibracion-fluke}}

\par 
Hay muchas disciplinas de calibración, cada una con diferentes tipos de calibradores y referencias de calibración. Para tener una idea de los tipos de calibradores e instrumentos que están disponibles. Las disciplinas comunes de calibración incluyen, pero no están limitadas a: eléctrica, radio frecuencia, temperatura, humedad, presión, flujo de aire, dimensional, tiempo.

\par \noindent
La disciplina de calibración en la cual estamos interesados para evaluar nuestro prototipo es la calibración de temperatura.
