\section{Introducción}

\par
De las muchas herramientas e instrumentos para el examen clínico, ninguno ha tenido aplicación tan extendida como el termómetro. En el tiempo de Hipócrates, solo la mano fue utilizado para detectar el calor o el frío del humano cuerpo, aunque la fiebre y los escalofríos eran conocidos como signos de procesos mórbidos. En la medicina alejandrina, el pulso se observó como un índice de enfermedad, reemplazando la evaluación cruda de la temperatura. En la Edad Media, los cuatro humores fueron asignados cualidades de calor, frío, seco y húmedo, y por lo tanto fiebre de nuevo adquirió importancia.\cite{intro-historia}

\par \noindent
Con el paso de los años inventores como Galileo Galilei fueron creando prototipos para una medición de la temperatura; sin embargo, muchos de estos prototipos no poseían una escala de temperatura definida. En el año 1665 Christian Huygens agregó una escala de temperatura extendiéndose desde el punto de congelación hasta el punto de ebullición del agua, la escala era el sistema centígrado original.\cite{intro-historia}

\par \noindent
Luego Gabriel Daniel Fahrenheit diseño una nueva escala de temperatura basada en la escala de temperatura del astrónomo danes Ole Romer. Fahrenheit descubrió que el mercurio era más útil que el agua para medir la temperatura; ya que se expandía y se contraía a los cambios de temperatura rápidamente. A pesar de la época, siglo XVIII, Fahrenheit fabricaba termómetros muy precisos y que variaban muy poco sus mediciones entre ellos.\cite{intro-historia}

\par \noindent
Los termómetros elaborados por Fahrenheit fueron adoptados por los ingleses y su escala de temperatura era el estándar durante la época colonial, siglo XIX y XX, pero En 1742, el astrónomo sueco Anders Celsius re introdujo la escala centígrada en práctica, pero a pesar de las mejoras en el diseño del termómetro, su uso permaneció en gran parte descuidado hasta fines del siglo XIX. Donde científicos encontraron más conveniente la escala de temperatura Celsius donde su punto de fusión y ebullición del agua son 0 y 100 grados; en vez de 32 y 212 respectivamente.\cite{intro-historia}

\par \noindent
Sin embargo el tamaño de los termómetros seguían siendo una gran desventaja, el mercurio al ser ingerido o inhalado por un ser humano es toxico. Por lo que se dieron avances para reemplazar el mercurio como sensor de temperatura y fabricar termómetros más compactos y más precisos. Entre los avances recientes en el diseño del termómetro incluyen digital, electrónico directo y predictivo, infrarrojo termómetros de oído, y matriz de puntos o cambio de fase termómetros. Pero ninguno es completamente libre de problemas.\cite{intro-historia}

\par \noindent
Teniendo en cuenta toda la historia del termómetro. Se debe investigar los elementos que se utilizaron para la elaboración de un termómetro; manteniendo las características de ser una herramienta portátil y que obtenga lecturas de temperatura de manera precisa y rápida.