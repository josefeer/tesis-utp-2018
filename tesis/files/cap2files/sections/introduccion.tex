\section{Introducción}
\par 
De las muchas herramientas e instrumentos considerados como
esencial para el examen clínico, ninguno ha tenido
aplicación tan extendida como el termómetro clínico.
En el tiempo de Hipócrates, solo la mano
fue utilizado para detectar el calor o el frío del humano
cuerpo, aunque la fiebre y los escalofríos eran conocidos como signos
de procesos mórbidos. En la medicina alejandrina, el
el pulso se observó como un índice de enfermedad, reemplazando
la evaluación cruda de la temperatura. En el
Edad Media, los cuatro humores fueron asignados
cualidades de calor, frío, seco y húmedo, y por lo tanto fiebre
de nuevo adquirió importancia.

\par \noindent
Galileo en 1592 ideó un prototipo medición de la temperatura bruta
, pero no tenía escala y
por lo tanto, no hay lecturas numéricas; Además, era
afectado por la presión atmosférica. Un gran paso
adelante fue logrado por Santorio (Sanctorio
Sanctorius) que inventó un termómetro de boca.

\par \noindent
Santorio (1561-1636) fue un fisiólogo italiano,
profesor en Padua. Él hizo cuantitativo
experimentos en temperatura, respiración y peso,
y midió la "transpiración insensible" que sentó
la base para el estudio del metabolismo. Santorino produjo varios diseños, pero todos fueron
engorrosos y requerían mucho tiempo para medir la temperatura oral.

\par \noindent
En 1665, Christiaan Huygens agregó una escala
extendiéndose desde el punto de congelación hasta el punto de ebullición
punto de agua, el sistema centígrado original.
Gabriel Daniel Fahrenheit basó su nueva escala
en una mezcla de hielo y cloruro de amonio como
el punto más bajo. Él encontró que el mercurio era más útil
que el agua, ya que se expandió y contrajo más
rápidamente.\cite{intro-historia}

\par \noindent
En 1742, el astrónomo sueco Anders Celsius
reintrodujo la escala centígrada en práctica, pero
a pesar de las mejoras en el diseño del termómetro,
su uso permaneció en gran parte descuidado hasta fines del siglo XIX
siglo.

\par \noindent
Sin embargo el tamaño de
los termómetros siguen siendo una gran desventaja.
Aitkin en 1852 hizo un instrumento de mercurio con un
tubo más angosto ubicado sobre un depósito de bulbo; esta
se aseguró de que el mercurio no retrocediera después de la
la lectura había sido tomada. Fue dejado a Thomas
Clifford Allbutt (1836-1925) para diseñar en 1866
un termómetro clínico de 6 pulgadas, convenientemente portátil, 4
capaz de registrar una temperatura en 5 min. Eso
reemplazó un modelo de un pie de largo, que requirió 20
minutos para determinar la temperatura de un paciente. los
medida de la temperatura se convirtio en un procedimiento indispensable para el diagnostico de pacientes.

\par \noindent
Los avances recientes en el diseño del termómetro incluyen
digital, electrónico directo y predictivo, infrarrojo
termómetros de oído, y matriz de puntos o cambio de fase
termómetros. Pero ninguno es completamente libre de
problemas.\cite{intro-historia}

\par \noindent
Teniendo en cuenta toda la historia del termómetro y que su aplicación en la medicina fue lo que la convirtió en una herramienta indispensable. Debemos tomar en cuenta los elementos que son utilizados hoy en día para la elaboración de un termómetro digital; manteniendo las caracteristicas de ser una herramienta pórtatil y que obtenga lecturas de temperatura de manera precisa y rápida.