\section{Tecnología Existente}

\par \noindent
En esta sección investigaremos sobre las tecnologías a nuestro alcance en el campo de la electrónica, desarrollo de software y diseño digital para el desarrollo de nuestro prototipo de termómetro. Empezaremos por el área de desarrollo de software, específicamente en el desarrollo de software móvil es decir software para teléfonos inteligentes o smartphones.

\subsection{Smartphone}

\par 
El teléfono inteligente (smartphone en inglés) es un tipo de ordenador de bolsillo que combina los elementos de una tablet con los de un teléfono celular. Sobre una plataforma informática móvil, con mayor capacidad de almacenar datos y realizar actividades, semejante a la de una minicomputadora, y con una mayor conectividad que un teléfono móvil convencional. El término inteligente, que se utiliza con fines comerciales, hace referencia a la capacidad de usarse como un computador de bolsillo, y llega incluso a reemplazar a una computadora personal en algunos casos\cite{smartphone}.
	
\par \noindent
Generalmente, los teléfonos con pantallas táctiles son los llamados teléfonos inteligentes, pero el soporte completo al correo electrónico parece ser una característica indispensable encontrada en todos los modelos existentes y anunciados desde 2007\cite{smartphone}.
		
\par \noindent
Entre otros rasgos comunes está la función multitarea, el acceso a Internet vía Wifi o redes 2G, 3G o 4G, función multimedia (cámara y reproductor de videos/mp3), a los programas de agenda, administración de contactos, acelerómetros, GPS y algunos programas de navegación, así como ocasionalmente la habilidad de y leer documentos de negocios en variedad de formatos como PDF y Microsoft Office\cite{smartphone}.

\par \noindent
Los sistemas operativos móviles más frecuentes utilizados por los teléfonos inteligentes son Android (de Google), iOS (de Apple) y Windows 10 (de Microsoft)\cite{smartphone}.

\begin{figure}[H]
	\includegraphics[width=\textwidth]{smartphone1.jpg}
	\caption{Smartphones con los Sistemas Operativos Móviles Mas Populares}
\end{figure}

\par \noindent
En nuestro trabajo utilizaremos exclusivamente para desarrollar el sistema operativo movil Android, debido a su popularidad en el mercado y su vasta documentación para desarrollar aplicaciones.

\subsection{Android}

\par
Android es un sistema operativo basado en el núcleo Linux. Fue diseñado principalmente para dispositivos móviles con pantalla táctil, como teléfonos inteligentes, tablets o tabléfonos; y también para relojes inteligentes, televisores y automóviles. Inicialmente fue desarrollado por Android Inc. la cual fue fundado en 2003 por Andy Rubin, Rich Miner, Nick Sears y Chris White con el objetivo de desarrollar \textquotedblleft dispositivos móviles que están al corriente de la ubicación y preferencias del usuario \textquotedblright\cite{android-wiki}. En un principio la intención era desarrollar un sistema operativo avanzado para cámaras digitales, pero más tarde se cambió el foco al determinar que el mercado de las cámaras digitales no era lo suficientemente grande. Se redirigirían los esfuerzos a crear un sistema que pudiera competir con Symbian y Windows Mobile\cite{android-xataka}. La versión básica de Android es conocida como Android Open Source Project (AOSP)\cite{android-wiki}. A continuación, una imagen de la evolución del logo de Android. 

\begin{figure}[H]
	\includegraphics[width=\textwidth]{android1.jpg}
	\caption{Diseño preliminar para el logo de Android (Izquierda) y el logo oficial (Derecha)}
\end{figure}

\par \noindent
A partir del año 2011 Google ha ido actualizando la versión de Android, todos los años. Hoy en día la última versión de Android es \textquotedblleft Android 8.0 Oreo \textquotedblright. Sin embargo, la versión de Android con más dispositivo o smartphones, en uso actualmente, es Android 5.0 Lollipop, la cual será presentada a continuación.

\par \noindent
A continuación se verán las tecnologías que se aplicarón en el área de electrónica.

\subsection{Arduino}

\par
Arduino es una plataforma de electrónica de código abierto basada en hardware y software fácil de usar. Las placas Arduino pueden leer entradas (luz en un sensor, un dedo en un botón o un mensaje de Twitter) y convertirlo en una salida, activar un motor, encender un LED y publicar algo en línea. Puede decirle a su placa qué hacer enviando un conjunto de instrucciones al microcontrolador en la placa. Para hacerlo, utiliza el lenguaje de programación Arduino (basado en \textquotedblleft Wiring\textquotedblright) y el software Arduino (IDE), basado en \textquotedblleft Processing \textquotedblright\cite{arduino-intro}.

\begin{figure}[H]
	\centering
	\includegraphics[width=5cm, height=4cm]{arduino1.png}
	\caption{Logo Oficial de Arduino}
\end{figure}

\par \noindent
Con los años, Arduino ha sido el cerebro de miles de proyectos, desde objetos cotidianos hasta complejos instrumentos científicos. Una comunidad mundial de fabricantes (estudiantes, aficionados, artistas, programadores y profesionales) se ha reunido en torno a esta plataforma de código abierto, sus contribuciones se han añadido a una increíble cantidad de conocimiento accesible que puede ser de gran ayuda para principiantes y expertos por igual\cite{arduino-intro}.

\begin{figure}[H]
	\centering
	\includegraphics[width=6cm, height=5cm]{arduino2.jpg}
	\caption{Placa Arduino Uno}
\end{figure}

\par \noindent
Arduino nació en el Ivrea Interaction Design Institute como una herramienta fácil para el prototipado rápido, dirigido a estudiantes sin experiencia en electrónica y programación. Tan pronto como llegó a una comunidad más amplia, la placa Arduino comenzó a cambiar para adaptarse a las nuevas necesidades y desafíos, diferenciando su oferta de simples placas de 8 bits para productos para aplicaciones IoT, wearable, impresión 3D y entornos integrados. Todos los tableros Arduino son completamente de código abierto, lo que permite a los usuarios construirlos de forma independiente y eventualmente adaptarlos a sus necesidades particulares. El software también es de código abierto y está creciendo a través de las contribuciones de los usuarios en todo el mundo\cite{arduino-intro}.

\begin{figure}[H]
	\centering
	\includegraphics[width=6cm, height=5cm]{arduino3.jpg}
	\caption{Placa Arduino Nano}
\end{figure}

\subsubsection{Componentes Pasivos}

\par \noindent
Los componentes pasivos principales son los siguientes:

\paragraph{Resistencia}
Son usados para establecer corrientes de operación y niveles de señal. Resistencias se utilizan en los circuitos de alimentación para reducir los voltajes al disipar la potencia, medir las corrientes y descargar los capacitores después de que se desconecta la energía. Una resistencia está hecha de elementos conductores (carbono, o una película delgada de metal o carbono, o un cable de baja conductividad), con un cable o contactos en cada extremo\cite{artofelectronics}. Se caracteriza por su resistencia y es definido por:

$$R = V/I$$

\begin{nscenter}
	Ley de Ohm: Donde R es de Resistencia, V de Voltaje e I de corriente.
\end{nscenter}

\begin{figure}[H]
	\centering
	\includegraphics[width=6cm, height=4cm]{resistor1.png}
	\caption{Ejemplo de Resistor y Simbología}
\end{figure}

\paragraph{Capacitor }
Un capacitor (el nombre antiguo era
condensador) es un dispositivo que tiene dos cables que sobresalen y tiene la propiedad\cite{artofelectronics} :
$$Q = CV$$ 

\begin{nscenter}
	Donde Q es carga almacenada en coulombs, C la capacitancia en faradios y V es el diferencia de potencia en voltios.
\end{nscenter}

\begin{figure}[H]
	\centering
	\includegraphics[width=8cm, height=6cm]{capacitor1.png}
	\caption{Ejemplo de Capacitor y Simbología}
\end{figure}

\par \noindent
Los capacitores esencialmente almacenan energía, pero es principalmente utilizado en corriente directa para filtrar picos de voltage provenientes de la fuente de la fuente de poder\cite{artofelectronics}.

\paragraph{Inductor}
Están estrechamente relacionados con condensadores: la tasa de cambio de corriente en un inductor es proporcional al voltaje aplicado a través de él (para un condensador es al revés, la tasa de cambio de voltaje es proporcional a la corriente a través de él) \cite{artofelectronics}. La ecuación de definición para un inductor es:
$$V = L\frac{dl}{dt}$$

\par \noindent
donde $L$ se llama inductancia y se mide en henrios
(o $mH$, $pH$, $nH$, etc.). Poniendo un voltaje constante a través de un
inductor hace que la corriente se eleve como una rampa (en comparación con un condensador, en el que una corriente constante causa el voltaje subir como una rampa); l $V$ a través de 1 $H$ produce una corriente que aumenta a 1 amperio por segundo \cite{artofelectronics}.

\begin{figure}[H]
	\centering
	\includegraphics[width=8cm, height=6cm]{inductor1.png}
	\caption{Ejemplo de Inductor y Simbología}
\end{figure}

\par \noindent
Al igual que con los capacitores, la energía invertida en el aumento de la corriente en un inductor se almacena internamente, aquí en la forma de campos magnéticos \cite{artofelectronics}.

\par \noindent
Ya definidos los componentes electricos que componen los modulos de arduino procedemos a explicar los dos modulos utilizados en este proyecto \cite{artofelectronics}.

\subsubsection{Módulos Arduino}

\par
Los módulos de Arduino son esencialmente placas de circuitos independientes que integran uno o múltiples circuitos integrados, sensores, pantallas LCD, componentes electrónicos y una interfaz de pines para una comunicación sencilla con la placa Arduino. Los módulos nos permiten agregar funcionalidad a nuestra placa arduino y son como piezas de un rompecabezas donde el resultado final es el prototipo deseado.

\paragraph{Modulo Bluetooth HC-05}
El módulo HC-05 es un módulo Bluetooth SPP (Serial Port Protocol por sus siglas en ingles) fácil de usar, diseñado para la configuración de conexión en serie inalámbrica transparente. El módulo Bluetooth HC-05 se puede utilizar en una configuración maestra o esclava, lo que la convierte en una excelente solución para la comunicación inalámbrica .Este puerto en serie del módulo bluetooth es completamente calificado Bluetooth V2.0 + EDR (Enhanced Data Rate) Modulación de 3Mbps con un completo transceptor de radio de 2.4GHz y banda de base\cite{bluetooth}.

\begin{figure}[H]
	\centering
	\includegraphics[width=8cm, height=6cm]{modulos1.jpg}
	\caption{Placa HC-05 Modelo ZS-040}
\end{figure}

\paragraph{Modulo Wi-Fi nRF24L01+}
El nRF24L01+ es un transceptor de 2.4GHz de un solo chip con un motor de protocolo de banda base integrado
, adecuado para aplicaciones inalámbricas de muy baja potencia. El nRF24L01 + está diseñado
para el funcionamiento en la banda de frecuencia ISM mundial a 2.400 - 2.4835GHz\cite{nrf}.

\par \noindent
Para diseñar un sistema de radio con nRF24L01 +, simplemente necesita una MCU (microcontrolador) y algunos componentes externos pasivos.

\par \noindent
El motor de protocolo de banda base integrado se basa en la comunicación por paquetes
y es compatible con varios modos, desde la operación manual hasta la operación de protocolo autónomo avanzado. 
Los FIFOs internos aseguran un flujo de datos sin problemas entre la interfaz de radio y la MCU del sistema. El motor mejorado reduce el costo del sistema mediante el manejo de todas las operaciones de capa de enlace de alta velocidad\cite{nrf}.

\begin{figure}[H]
	\centering
	\includegraphics[width=8cm, height=6cm]{modulos2.jpg}
	\caption{Placa nRF24L01+}
\end{figure}

\par \noindent
El modulo nRF24L01+ admite una velocidad de datos de aire de 250 kbps, 1 Mbps y 2 Mbps.
La alta velocidad de datos de aire combinada con dos modos de ahorro de energía hacen que el nRF24L01+ sea muy adecuado para ultra bajo
diseños de energía\cite{nrf}.

\par \noindent
Nuestro prototipo debe ser capaz de visualizar la información capturada y enviarla a nuestra aplicación movil. 
Los modulos previamente mencionados son los encargados de la comunicación inalambrica y una pantalla LCD es la encargada de reflejar la información en tiempo real al usuario final; no obstante, primero debemos tener una lista de pantallas LCD cantidatas.

\subsubsection{Pantallas LCD}

Una pantalla LCD (Liquid Crystal Display, en inglés) es una pantalla delgada y plana formada por un número de píxeles en color o monocromos colocados delante de una fuente de luz o reflectora. A menudo se utiliza en dispositivos electrónicos de pilas, ya que utiliza cantidades muy pequeñas de energía eléctrica\cite{lcd}. Adicional las pantallas LCD deben utilizar un controlador, usualmente un circuito integrado, para poder comunicarse con las placas arduino. 

\paragraph{LCM-1602}

La pantalla LCM-1602 es una pantalla clasica en proyecto de arduino. El LCM-1602 cuenta con una luz de fondo LED de color amarillo y requiere de una conexión de 5V, que puede proporcionar la mayoría de las placas arduino. Lastimosamente solamente es capaz de visualizar 16 caracteres monocromaticos por línea, la pantalla cuenta con 2 líneas, reflejando un total de solamente 32 caracteres por líneas\cite{lcm}. Se recomienda utilizar un controlador externo para poder conectar la pantalla con arduino; sino, se requerirían mucho mas conexiónes fisicas para la conexíon con arduino, ver figura 2.12. Las medidas fisicas de esta pantalla, hace dificil implementarla en prototipos donde el espacio esta comprometido.

\begin{figure}[H]
	\centering
	\includegraphics[width=0.5\linewidth]{lcd1.jpg}
	\caption{Pantalla LCM-1602, junto con IC para mejor integración con arduino.}
\end{figure}

\paragraph{NOKIA 5110}
Esta pantalla LCD es la misma utilizada por el celular Nokia 5110 o Nokia 3310, ver figura 2.13 , de allí su nombre\cite{nokia1}. 
Sin embargo Phillips, competencia de Nokia hace años atras, es el fabricante del controlador de esta pantalla LCD, el PCD8544. Phillips indica que esta pantalla tiene una resolución de 48x84 pixeles; siendo esta pantalla mas pequeña que el LCM-1602 pero con mayor capacidad de visualizar caracteres\cite{nokia2}. Debe ser alimentado por 3.3V exclusivamente, de ser alimentado con 5V se daña el controlador, y viene integrado con luces de fondo LED\cite{nokia2}. El unico incoveniente de esta pantalla viene siendo su tamaño compacto; ya que, la visualización de la información es muy pequeña.

\begin{figure}[H]
	\centering
	\includegraphics[width=0.35\linewidth]{lcd2.jpg}
	\caption{LCD Nokia 5110 con placa para una mejor intregración con arduino.}
\end{figure}

\paragraph{LCD TFT ILI9341}
La pantalla LCD ILI9341 es llamado después del controlador que es utilizado para poder comunicarse con ella. Tiene versiones de 2.2, 2.4, 2.8 y 3.2 pulgadas. La pantalla ILI9341 debe ser alimentada con 5V, junto con la luz de fondo led; no obstante, el controlador utiliza una logica de 3.3V; tiene una resolución de 240x320 pixeles y es una pantalla multicolor, excelente para dispositivo de tamaño pequeño o mediano donde una larga duración de la bateria es indispensable\cite{ili9341}. 

\begin{figure}[H]
	\centering
	\includegraphics[width=0.9\linewidth]{lcd3.jpg}
	\caption{ILI9341 de 2.8 pulgadas con placa para mejor integración con arduino.}
\end{figure} 

\par \noindent
Las pantalla LCD es importante, en nuestro prototipo, porque son el principal medio de comunicación entre el prototipo y el usuario final. El ultimo punto es que el prototipo debe ser capaz de capturar mediciones de temperatura de manera precisa; por ende se requiere un sensor eficaz y eficiente.




\subsection{Sensores en General}

\par
Un sensor es un dispositivo capaz de detectar magnitudes físicas o químicas, llamadas variables de instrumentación, y transformarlas en variables eléctricas.

\begin{itemize}
	\item Las variables de instrumentación pueden ser por ejemplo: temperatura, intensidad lumínica, distancia, aceleración, inclinación, desplazamiento, presión, fuerza, torsión, humedad, movimiento, pH, etc.
	
	\item Una variable eléctrica puede ser una resistencia eléctrica (como en una RTD), una capacidad eléctrica (como en un sensor de humedad o un sensor capacitivo), una tensión eléctrica (como en un termopar), una corriente eléctrica (como en un fototransistor), etc.
\end{itemize}

\par \noindent
Los sensores se pueden clasificar en función de los datos de salida en: digitales y analógicos\cite{sensores-arduino}.

\subsubsection{Características de los Sensores \cite{sensores-wiki}}

\begin{itemize}
	
	\item Rango de medida: dominio en la magnitud medida en el que puede aplicarse el sensor.
	
	\item Precisión: es el error de medida máximo esperado.
	
	\item Offset o desviación de cero:  valor de la variable de salida cuando la variable de entrada es nula. Si el rango de medida no llega a valores nulos de la variable de entrada, habitualmente se establece otro punto de referencia para definir el offset.
	
	\item Linealidad o correlación lineal.
	
	\item Sensibilidad de un sensor: suponiendo que es de entrada a salida y la variación de la magnitud de entrada.
	
	\item Resolución: mínima variación de la magnitud de entrada que puede detectarse a la salida.
	
	\item Rapidez de respuesta: puede ser un tiempo fijo o depender de cuánto varíe la magnitud a medir. Depende de la capacidad del sistema para seguir las variaciones de la magnitud de entrada.
	
	\item Derivas: son otras magnitudes, aparte de la medida como magnitud de entrada, que influyen en la variable de salida. Por ejemplo, pueden ser condiciones ambientales, como la humedad, la temperatura u otras como el envejecimiento (oxidación, desgaste, etc.) del sensor.
	
	\item Repetitividad: error esperado al repetir varias veces la misma medida.
	
\end{itemize}

\par \noindent
Debido a que solamente la magnitud física que nos interesa para nuestro proyecto es la temperatura. Tomamos encuenta los posibles candidatos para utilizar como sensor de temperatura en el prototipo.

\clearpage

\subsubsection{Sensores de Temperatura}	

\paragraph{DS18B20}
El termómetro digital DS18B20 proporciona de 9 bits a 12 bits.
Mediciones de temperatura en grados celsius y tiene una función de alarma no volátil programable por el usuario. El DS18B20 se comunica a través de un Bus de 1 cable que por definición requiere solo una línea de datos (y tierra) para la comunicación con un microprocesador central. Además, el DS18B20 puede obtener potencia directamente desde la línea de datos, eliminando la necesidad de una fuente de alimentación externa\cite{ds18b20}.

\begin{figure}[H]
	\centering
	\includegraphics[width=0.5\textwidth]{sensores1.jpg}
	\caption{Sensor DS18B20 estilo sonda}
\end{figure}

\par \noindent
Mide las temperaturas de -55 ° C a 125 ° C, ± 0.5 ° C de  precisión entre -10 ° C a + 85 ° C y resolución programable de 9 a 12 bits.

\par \noindent
Las aplicaciones que pueden beneficiarse de esta característica incluyen
Controles ambientales, monitoreo de temperatura
sistemas dentro de edificios, equipos o maquinaria, y
sistemas de monitoreo y control de procesos\cite{ds18b20}. 


\paragraph{DHT11}
El sensor digital de temperatura y humedad DHT11 es un sensor compuesto que contiene una
señal digital de salida de la temperatura y la humedad. Aplicación de módulos digitales dedicados
tecnología de recolección y la tecnología de detección de temperatura y humedad, para asegurar que
el producto tiene una alta fiabilidad y una excelente estabilidad a largo plazo. El sensor incluye un sentido resistivo
de componentes húmedos y dispositivos de medición de temperatura NTC, y conectado con un
microcontrolador de alto rendimiento de 8 bits\cite{dht11}.

\begin{figure}[H]
	\centering
	\includegraphics[width=0.25\textwidth]{sensores2.jpg}
	\caption{Sensor DHT11 en placa}
\end{figure}

\par \noindent
Mide temperaturas de 0 ° C a 50 ° C con una precisión de ±2 ° C en 25 ° C y una resoluación no programable de 16 bits.

\par \noindent
Aplicaciones de este sensor son principalmente para mediciones de humedad y temperatura relativa. Donde se require obtener valores en ambas magnitudes y no tanta precisión de una magnitud en particular\cite{dht11}. 

\par \noindent
Los sensores de temperatura y la plataforma de Arduino nos permite crear un prototipo de medición de temperatura; sin embargo, aun no es un termómetro como tal y mucho menos uno que SIGCSA pueda utilizar en campo. Hay que definir el concepto de termómetro para aplicarlo en nuestro prototipo.



\subsection{Impresión 3D}

\par
Es necesario definir el concepto de prototipo porque es el punto de partida para el desarrollo de la tecnología de Prototipado Rapido y consecuentemente, para las impresoras tridimensionales. Podemos definir como prototipo al primer ejemplar que se fabrica de una figura, un invento o elemento físico, y sirve de modelo para fabricar otros iguales.\cite{impresoras3d-valverde}

\par \noindent
Los prototipos tiene el propósito de probar suposiciones formuladas en busca de una solución a un problema determinado. Ademas se convierten en un diseño enfocado al usuario, donde es necesario un proceso interactivo entre el propio diseñador y el consumidor. Así mismo, todos los prototipos son objetos de desarrollo de bajo costo, pero en muchas ocasiones la necesidad de elaborar varios prototipos o de realizar un prototipo con una estética cuidada, eleva los costes de producción. Por lo tanto, el alto coste de producción de prototipos y tiempo de ejecución de los mismos puede suponer un problema. Los sistemas de Prototipado Rápido surgen con la necesidad de solventar estos problemas en el uso de prototipos, se busca por lo tanto, una manera de elaborarlos con un aspecto cuidado y atractivo para el usuario, con un método de fabricación rápido y fácil de modificar, económicos y que pueden ser probados por el consumidor. Por consiguiente, esta herramienta resultará útil y funcional. Rápidamente estos sistemas de construcción aditiva partirán del desarrollo tecnológico de maquinara destinada a uso industrial\cite{impresoras3d-valverde}. 

\begin{figure}[H]
	\centering
	\includegraphics[width=0.5\textwidth]{impresion1.jpg}
	\caption{Ciclo del Prototipado Rapido}
\end{figure}

\par \noindent
El Prototipado Rápido se convierte así, en un proceso utilizado para la fabricación de prototipos, los cuales, como hemos visto, son objetos que no estan diseñados a uso final, sino más bien a modo de prueba de diseño\cite{impresoras3d-valverde}.

\par \noindent
La impresión 3D, también conocida como impresión tridimensional, es un método para
producir objetos tridimensionales con un aparato tecnológico al colocar varias capas
bidimensionales de cierto material una sobre la otra. El proceso es similar a la impresión
bidimensional en la cual se coloca tinta sobre papel, sin embargo, en la impresión
tridimensional se coloca comúnmente plástico sobre una superficie que nos permite fabricar objetos de tres
dimensiones; sin embargo, tambien se utilizan otros materiales como metales, arcilla y resinas. Esta tecnología se utiliza principalmente en la producción de prototipos durante
el diseño de algún nuevo producto.\cite{impresoras3d-monterrey}

\par \noindent
Durante los últimos años, la tecnología de impresión tridimensional ha avanzado de
manera exponencial. Este tipo de impresoras se han vuelto cada vez más accesibles y por ello
ahora es común encontrarlas en fábricas, industrias, instituciones educativas, instituciones de
investigación e inclusive para uso personal\cite{impresoras3d-monterrey}.

\subsubsection{Técnicas de Impresión 3D}

\par \noindent
Hablaremos de las principales técnicas comerciales de impresión 3D; ya que, la impresión 3D es una tecnología de hace muchos años. No fue hasta a mediados de la década del 2000 que la impresoras 3D comenzaron a bajar sus precios y apuntados a un mercado menos especializado.

\begin{figure}[H]
	\centering
	\includegraphics[width=0.8\textwidth]{impresion2.png}
	\caption{Principales Técnicas Comerciales de Impresion 3D}
\end{figure}

\paragraph{FDM (Fused Deposition Modeling)}
Consiste en la deposición por capas de material normalmente construido por filamentos de polímeros termoplásticos, que se funden y se extruyen por medio de una boquilla, solidificándose cuando salen de dicha boquilla al exterior, ver figura 2.18. (FDM).\cite{impresoras3d-valverde}.

\begin{figure}[H]
	\centering
	\includegraphics[width=0.5\textwidth]{impresion3.jpg}
	\caption{Impresora FDM: MP Select Mini V2 de Monoprice.}
\end{figure}

\clearpage

\paragraph{SLA (Estereografía)}
Se basa en las propiedades de una resina fotosensible que se solidifica mediante la proyección de un láser (UV) de frecuencia y potencia concreta, ver figura 2.18. (SLA)\cite{impresoras3d-valverde}.

\begin{figure}[H]
	\centering
	\includegraphics[width=0.3\textwidth]{impresion4.jpg}
	\caption{Impresora SLA: Form2 de Formlabs.}
\end{figure}

\paragraph{DLP (Procesamiento de Luz Digital)}
En este proceso, una cubeta de polímero líquido se expone a la luz de un proyector DLP en condiciones de seguridad. El proyector DLP muestra la imagen del modelo 3D en el polímero líquido, ver figura 2.18 (DLP). El polímero líquido expuesto se endurece y la placa de construcción se mueve hacia abajo y el polímero líquido queda una vez más expuesto a la luz. El proceso se repite hasta que el modelo 3D se completa y la tina se drena de líquido, lo que revela el modelo solidificado\cite{impresoras3d-blog}.

\begin{figure}[H]
	\centering
	\includegraphics[width=0.5\textwidth]{impresion5.jpg}
	\caption{Impresora DLP: LCD Photon de Anycubic.}
\end{figure}



\par \noindent
Una vez realizada la investigación de las tecnologías existentes para el desarrollo de nuestro prototipo. Debemos recordar que el prototipo debe cumplir con un estándar para que la compañía SIGCSA pueda utilizarlo en campo y para ello el prototipo debe estar calibrado. Por ende, es necesario investigar sobre la disciplina de la calibración.
