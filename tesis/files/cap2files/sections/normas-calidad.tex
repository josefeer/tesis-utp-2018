\section{Normas de Calidad}

\par 
En la sección anterior hablamos sobre lo importante que es la calibración de los instrumentos de medición de temperatura. Hablamos de entidades gubernamentales que se encargan de estipular ciertos estándares; sin embargo, estas entidades de ciertos países como Estados Unidos, Francia y Reino Unido trabajan en conjunto con la ISO (Organización Internacional de Normalización) para que estos estándares sean aplicados en todo el mundo. ISO es una federación mundial de organismos nacionales de normalización (organismos miembros de ISO). El trabajo de preparación de las normas internacionales normalmente se realiza a través de los comités técnicos de ISO. Adicional ISO colabora estrechamente con la Comisión Electrotécnica Internacional (IEC) en todas las materias de normalización electrotécnica.

\par \noindent
Este proyecto requiere el conocimiento de una norma internacional. La norma internacional en cuestión es la ISO 17025:2005 "Requisitos generales para la competencia de los laboratorios de prueba y calibración". Un requisito para la interpretación correcta de esta norma es el conocimiento de la norma ISO 9000:2015 "Sistemas de gestión de calidad: Fundamentos y Vocabulario". 

\subsection{ISO 9000:2015}
	\subsubsection{Generalidades}
		\par
			Los conceptos y los principios de la gestión de la calidad descritos en esta Norma Internacional
			proporcionan a la organización la capacidad de cumplir los retos presentados por un entorno que es
			profundamente diferente al de décadas recientes El contexto en el que trabaja una organización
			actualmente se caracteriza por el cambio acelerado, la globalización de los mercados, los recursos
			limitados y la aparición del conocimiento como un recurso principal El impacto de la calidad se extiende
			más allá de la satisfacción del cliente puede tener además un impacto directo en la reputación de la
			organización.
		
		\newpage
		\thispagestyle{plain}
	
		\par 
			\noindent La sociedad está más formada y demanda más. lo que hace a las parles interesadas más influyentes
			progresivamente. Esta Norma Internacional proporciona una manera de pensar más amplia en relación
			con la organización, proporcionando conceptos y principios fundamentales para utilizar en el desarrollo
			de un Sistema de Gestión de la Calidad (SGC).
			
		\par \noindent 
			Todos los conceptos, principios y sus interrelaciones deberían verse como un conjunto y no aislados unos de otros. Un concepto o principio individual no es más importante que otro. En cada momento es crítico encontrar un balance correcto en su aplicación.
			
	\subsubsection{Conceptos Fundamentales}
	
		\paragraph{Calidad}
			Una organización orientada la calidad promueva una cultura que da como resultado comportamientos,
			actitudes, actividades y procesos para proporcionar valor mediante el cumplimiento de las necesidades y
			expectativas de los clientes y otras partes interesadas pertinentes.
			
		\par \noindent 
			La calidad de los productos y servicios de una organización está determinada por la capacidad para
			satisfacer a los clientes, y por el impacto previsto y el no previsto sobre las partes interesadas
			pertinentes.
			
		\par \noindent 
			La calidad de los productos y servicios incluye no solo su función y desempeño previstos, sino también su valor percibido y el beneficio para el cliente.
		
		\paragraph{Sistemas de gestión de calidad}
			Un SGC comprende actividades mediante las que la organización identifica sus objetivos y determina los
			procesos y recursos requeridos para lograr los resultados deseados.
			El SGC gestiona los procesos que interactúan y los recursos que se requieren para proporcionar valor y
			lograr los resultados para las partes interesadas pertinentes.
			
		\par \noindent 
			EL SGC posibilita a la alta dirección optimizar el uso de los recursos considerando las consecuencias de sus decisiones a largo y corto plazo.
			
		\newpage
		\thispagestyle{plain}
			
		\paragraph{Contexto de una organización}
			Comprender el contexto de una organización es un proceso Este proceso determina los factores que
			influyen en el propósito, objetivos y sostenibilidad de la organización. Considera factores internos tales
			como los valores, cultura, conocimiento y desempeño de la organización. También considera factores
			externos tales como entornos legales, tecnológicos, de competitividad. de mercados, culturales, sociales
			y económicos.
			
			
		
		\par \noindent 
			La visión, misión, políticas y objetivos son ejemplos de las formas en las pueden expresar los propósitos
			de la organización.
			
		\paragraph{Partes interesadas}
			El concepto de partes interesadas se extiende más allá del enfoque únicamente al cliente. Es importante
			considerar todas las partes interesadas pertinentes.
			
		\par \noindent 
			Parte del proceso para la comprensión del contexto de la organización es identificar sus partes
			interesadas. Las partes interesadas pertinentes son aquellas que generan riesgo significativo para la
			sostenibilidad de la organización si sus necesidades y expectativas no se cumplen. Las organizaciones
			definen qué resultados son necesarios para proporcionar a aquellas partes interesadas pertinentes para
			reducir dicho riesgo
			
		\par \noindent 
			Las organizaciones atraen, consiguen y conservan el apoyo de las partes interesadas pertinentes de las
			que dependen para su éxito
				
	\subsubsection{Principios de la gestión de la calidad}
		\paragraph{Enfoque al cliente}
			El enfoque principal dela gestión de la calidad es cumplir con los requisitos del cliente y tratar de exceder
			las expectativas del cliente.
		\paragraph{Liderazgo}
			Los líderes en todos los niveles establecen la unidad de propósito y la dirección y crean condiciones en
			las que las personas se implican en el logro de los objetivos de la calidad de la organización.
			
		\newpage
		\thispagestyle{plain}
		
		\paragraph{Compromiso de las personas}
			Las personas competentes, empoderadas y comprometidas en toda la organización son esenciales para
			aumentar la capacidad de la organización para generar y proporcionar valor.
		\paragraph{Enfoque a procesos}
			Se alcanzan resultados coherentes y previsibles de manera mas eficaz y eficiente cuando las actividades
			se entienden y gestionan como procesos interrelacionados que funcionan como un sistema coherente.
		\paragraph{Mejora}
			Las organizaciones con éxito tienen un enfoque continuo hacia la mejora.
		
		\paragraph{Toma de decisiones basada en evidencia}
			Las decisiones basadas en el análisis y la evaluación de datos e información tienen mayor probabilidad
			de producir los resultados deseados.
		
		\paragraph{Gestión de relaciones}
			Para el éxito sostenido, las organizaciones gestionan sus relaciones con las partes interesadas
			pertinentes, tales como los proveedores.
			
		\newpage
		\thispagestyle{plain}
	
	\subsubsection{Términos y definiciones}
		\paragraph{Términos relativos a la persona o personas}
			\begin{itemize}
				\item Alta Dirección: Persona o grupo de personas que dirige y controla una organización al más alto nivel.
				
				\item Participación Activa: Tomar parte en una actividad, evento o situación.
				
				\item Compromiso: Participación Activa en, y contribución a, las actividades para lograr objetivos compartidos
			\end{itemize}
		
		\paragraph{Términos relativos a la organización}
			\begin{itemize}
				\item Organización: Persona o grupo de personas que tiene sus propias funciones con responsabilidades, autoridades y
				relaciones para lograr sus objetivos.
				
				\item Contexto de la organización: Combinación de cuestiones internas y externas que pueden tener un efecto en el enfoque de la
				organización para el desarrollo y logro de sus objetivos.
				
				\item Parte interesada: Persona u organización que puede afectar, verse afectada o percibirse como afectada por una
				decisión o actividad.
				
				\item Cliente: Persona u organización  que podría recibir o que recibe un producto o un servicio
				destinado a esa persona u organización o requerido por ella.
				
				\item Proveedor: Organización que proporciona un producto o un servicio
				
				\item Proveedor externo: Proveedor que no es parte de la organización.
				
				\item Proveedor de PRC: (Proveedor de un proceso de resolución de conflictos)
				Persona u organización que provee y opera un proceso de resolución de conflictos externo.
				
				\item Asociación:  organización formada por organizaciones o personas miembro. 
				
				\item Función metrológica: Unidad funcional con responsabilidad administrativa y técnica para definir e implementar el sistema de
				gestión de las mediciones.
				
			\end{itemize}
		
		\newpage
		\thispagestyle{plain}
		
		\paragraph{Términos relativos a la actividad}
			\begin{itemize}
				\item Mejora: Actividad para mejorar el desempeño.
				
				\item Mejora continua: Actividad recurrente para mejorar el desempeño.
				
				\item Gestión: Actividades coordinadas para dirigir y controlar una organización.
				
				\item Gestión de calidad: Gestión con respecto a la calidad.
				
				\item Planificación de calidad: parte de la gestión de la calidad orientada a establecer los objetivos de la calidad y a la
				especificación de los procesos operativos necesarios y de los recursos relacionados para lograr
				los objetivos de la calidad.
				
				\item Aseguramiento de la calidad: Parte de la gestión de la calidad orientada a proporcionar confianza en que se cumplirán los
				requisitos de la calidad.
				
				\item Control de calidad: Parte de la gestión de la calidad orientada al cumplimiento de los requisitos de la calidad
				
				\item Mejora de calidad: Parte de la gestión de la calidad orientada a aumentar la capacidad de cumplir con los
				requisitos de la calidad.
				
				\item Gestión de la configuración: Actividades coordinadas para dirigir y controlar la configuración.
				
				\item Actividad: el menor objeto de trabajo identificado en un proyecto. 
				
				\item Gestión de proyectos: Planificación, organización, seguimiento , control e informe de todos los aspectos de unproyecto y la motivación de todos aquellos que están involucrados en él para alcanzar los objetivos del proyecto.
				
			\end{itemize}
		
		\paragraph{Términos relativos al proceso}
			\begin{itemize}
				\item Proceso: Conjunto de actividades mutuamente relacionadas que utilizan las entradas para proporcionar un
				resultado previsto.
				
				\item Proyecto: Proceso único, consistente en un conjunto de actividades coordinadas y controladas con fechas
				de inicio y de finalización, llevadas a cabo para lograr un objetivo conforme con requisitos específicos, incluyendo las limitaciones de tiempo, costo y recursos.
				
				\item Procedimiento: Forma especificada de llevar a cabo una actividad o un proceso.
				
				\item Contrato: Acuerdo vinculante.
				
\newpage
\thispagestyle{plain}
				
				\item Diseño y Desarrollo: Conjunto de procesos que transforman los requisitos para un objeto en requisitos más detallados para ese objeto.
			\end{itemize}
		
		\paragraph{Terminos relativos al sistema}
			\begin{itemize}
				\item Sistema: Conjunto de elementos interrelacionados o que interactúan.
				
				\item Infraestructura: sistema de instalaciones, equipos y servicios necesarios para el
				funcionamiento de una organización.
				
				\item Sistema de gestión: Conjunto de elementos de una organización interrelacionados o que interactúan para establecer políticas, objetivos y procesos para lograr estos objetivos.
				
				\item Sistema de gestión de calidad: Parte de un Sistema de Gestión relacionada con la calidad.
				
				\item Ambiente de trabajo: Conjunto de condiciones bajo las cuales se realiza el trabajo.
				
				\item Confirmación metrológica: Conjunto de operaciones necesarias para asegurarse de que el equipo de medición es
				conforme con los requisitos para su uso previsto. 
				
				\item Sistema de gestión de las mediciones: Conjunto de elementos interrelacionados, o que interactúan, necesarios para lograr la confirmación
				metrológica y el control de los procesos de medición.
				
				\item Política: intenciones y dirección de una organización, como las expresa formalmente
				su alta dirección.
				
				\item Política de calidad: Política relativa a la calidad. 
				
				\item Visión: aspiración de aquello que una organización querría llegar a ser, tal como lo expresa la alta dirección.
				
				\item Misión: propósito de la existencia de la organización, tal como lo expresa la alta dirección.
				
				\item Estrategia: Plan para lograr un objetivo a largo plazo o global.
			\end{itemize}
		
\newpage
\thispagestyle{plain}
		
		\paragraph{Términos relativos a los requisitos}
			\begin{itemize}
				\item Objeto, entidad, item: Cualquier cosa que puede percibirse o concebirse.
				
				\item Calidad: Grado en el que un conjunto de características inherentes de un objeto cumple con los requisitos.
				
				\item Clase: Categoría o rango dado a diferentes requisitos para un objeto que tienen el mismo uso funcional.
				
				\item Requisito: Necesidad o expectativa establecida, generalmente implícita u obligatoria.
				
				\item No conformidad: Incumplimiento de un requisito.
				
				\item Defecto: No conformidad relativa a un uso previsto o especificado.
				
				\item Conformidad: Cumplimiento de un requisito.
				
				\item Capacidad: Aptitud de un objeto para realizar una salida que cumplirá los requisitos para esa salida.
				
				\item Trazabilidad: Capacidad para seguir el histórico, la aplicación o la localización de un objeto.
				
				\item Confiabilidad: Capacidad para desempeñar cómo y cuándo se requiera.
				
				\item Innovación: Objeto nuevo o cambiado que crea o redistribuye valor.
			\end{itemize}
		
		\paragraph{Términos relativos al resultado}
		\begin{itemize}
			\item Objetivo: Resultado a lograr.
			
			\item Éxito: Logro de un objetivo.
			
			\item Salida: Resultado de un proceso.
			
			\item Producto: Salida de una organización que puede producirse sin que se lleve a cabo ninguna transacción entre la organización y el cliente.
			
			\item Servicio: Salida de una organización con al menos una actividad, necesariamente llevada a cabo
			entre la organización y el cliente.
			
			\item Desempeño: Resultado medible.
			
			\item Riesgo: Efecto de la incertidumbre.
			
			\item Eficiencia: Relación entre el resultado alcanzado y los recursos utilizados.
			
\newpage
\thispagestyle{plain}
			
			\item Eficacia: Grado en el que se realizan las actividades planificadas y se logran los resultados planificados.
			
		\end{itemize}
	
		\paragraph{Términos relativos a los datos, la información y la documentación}
		\begin{itemize}
			\item Datos: Hechos sobre un objeto.
			
			\item Información: Datos que poseen significado.
			
			\item Evidencia objetiva: Datos que respaldan la existencia o veracidad de algo.
			
			\item Sistema de información: <sistema de gestión de la calidad> red de canales de comunicación utilizados dentro de
			una organización.
			
			\item Documento: Información y el medio en el que está contenida.
			
			\item Información documentada: Información que una organización tiene que controlar y mantener, y el medio que la contiene.
			
			\item Especificación: Documento que establece requisitos.
			
			\item Manual de calidad: Especificación para el sistema de gestión de la calidad de una organización.
			
			\item Plan de calidad: Especificación de los procedimientos y recursos asociados a aplicar, cuándo deben aplicarse y quién debe aplicarlos a un objeto específico.
			
			\item Registro: Documento que presenta resultados obtenidos o proporciona evidencia de actividades realizadas.
			
			\item Verificación: Confirmación, mediante la aportación de evidencia objetiva de que se han cumplido
			los requisitos especificados.
			
			\item Validación: Confirmación, mediante la aportación de evidencia objetiva, de que se han cumplido los requisitos
			para una utilización o aplicación específica prevista.
			
			\item Caso específico: <plan de la calidad> tema del plan de la calidad.
			
\newpage
\thispagestyle{plain}
			
		\end{itemize}
	
		\paragraph{Términos relativos al cliente}
			\begin{itemize}
				\item Opiniones, comentarios y muestras de interés por un producto, un
				servicio o un proceso de tratamiento de quejas.
				
				\item Satisfacción del cliente: Percepción del cliente sobre el grado en que se han cumplido las expectativas de los clientes.
				
				\item Queja: expresión de insatisfacción hecha a una organización, relativa a
				su producto o servicio, o al propio proceso de tratamiento de quejas, donde
				explícita o implícitamente se espera una respuesta o resolución.
				
				\item Servicio al cliente: interacción de la organización con el cliente a lo largo del ciclo de vida de un producto o un servicio.
				
				\item Conflicto: desacuerdo, que surge de una queja presentada a un proveedor de PRC. 
			\end{itemize}
		
		\paragraph{Térmicos relativos a las características}
			\begin{itemize}
				\item Característica: Rasgos diferenciador.
				
				\item Característica de calidad: Característica inherente a un objeto relacionada con un requisito.
				
				\item Factor humano: Característica de una persona que tiene un impacto sobre un objeto bajo consideración.
				
				\item Competencia: Capacidad para aplicar conocimientos y habilidades con el fin de lograr los resultados previstos.
				
				\item Característica metrológica: Característica que puede influir sobre los resultados de la medición.
				
				\item Configuración: Características funcionales y físicas interrelacionadas de un producto o servicio
				definidas en la información sobre configuración del producto.
			\end{itemize}
		
		\paragraph{Términos relativos a las determinaciones}
			
			\begin{itemize}
				\item Determinación: Actividad para encontrar una o más características y sus valores característicos.
				
				\item Revisión: Determinación de la conveniencia, adecuación o eficacia de un objeto para lograr unos objetivos establecidos.
				
\newpage
\thispagestyle{plain}
				
				\item Seguimiento: Determinación del estado de un sistema, un proceso, un producto, un
				servicio o una actividad.
				
				\item Medición: Proceso para determinar un valor.
				
				\item Proceso de medición: Conjunto de operaciones que permiten determinar el valor de una magnitud.
				
				\item Equipo de medición: Instrumento de medición, software, patrón de medición, material de referencia o equipos auxiliares o
				combinación de ellos necesarios para llevar a cabo un proceso de medición.
				
				\item Inspección: Determinación de la conformidad con los requisitos  especificados.
				
				\item Ensayo: Determinación de acuerdo con los requisitos para un uso o aplicación previsto
				específico.
			\end{itemize}
		
		\paragraph{Términos relativos a las acciones}
			\begin{itemize}
				\item Acción preventiva: Acción tomada para eliminar la causa de una no conformidad potencial u otra situación potencial
				no deseable.
				
				\item Acción correctiva: Acción para eliminar la causa de una no conformidad y evitar que vuelva a ocurrir.
				
				\item Corrección: Acción para eliminar una no conformidad detectada.
				
				\item Reclasificación: variación de la clase de un producto o servicio no conforme para hacerlo
				conforme a requisitos diferentes de los requisitos iniciales.
				
				\item  Concesión: Autorización para utilizar o liberar un producto o servicio que no es conforme con los requisitos especificados.
				
				\item Permiso de desviación: Autorización para apartarse de los requisitos originalmente especificados de un producto 
				o servicio, antes de su realización.
				
				\item Liberación: Autorización para proseguir con la siguiente etapa de un proceso o el proceso siguiente.
				
				\item Reparación: Acción tomada sobre un producto o servicio no conforme para convertirlo en aceptable para su utilización prevista.
				
				\item Desecho: Acción tomada sobre un producto o servicio no conforme para impedir su uso
				inicialmente previsto.
				
			\end{itemize}
		
\newpage
\thispagestyle{plain}

\subsection{ISO 17025:2005}
\subsection*{Requisitos generales para la competencia de los laboratorios de ensayo y de calibración}

\par 
La primera edición (1999) de esta Norma Internacional fue producto de la amplia experiencia adquirida en la
implementación de la Guía ISO/IEC 25 y de la Norma EN 45001, a las que reemplazó. Contiene todos los
requisitos que tienen que cumplir los laboratorios de ensayo y de calibración se desean demostrar que poseen
un sistema de gestión, técnicamente competentes y capaces de generar resultados técnicamente
válidos. Es conveniente que los organismos de acreditación que reconocen la competencia de los laboratorios de
ensayo y de calibración se basen en esta Norma Internacional para sus acreditaciones.\cite{iso17025}

\subsubsection{Objeto y campo de aplicación}
\par
Esta Norma Internacional establece los requisitos generales para la competencia en la realización de
ensayos o de calibraciones, incluido el muestreo. Cubre los ensayos y las calibraciones que se realizan
utilizando métodos normalizados, métodos no normalizados y métodos desarrollados por el propio laboratorio.

\par \noindent
Esta Norma Internacional es aplicable a todas las organizaciones que realizan ensayos o calibraciones.
Éstas pueden ser, por ejemplo, los laboratorios de primera, segunda y tercera parte, y los laboratorios en los que los ensayos o las calibraciones forman parte de la inspección y la certificación de productos.

\par \noindent
Esta Norma Internacional es aplicable a todos los laboratorios, independientemente de la cantidad de
empleados o de la extensión del alcance de las actividades de ensayo o de calibración. Cuando un laboratorio no realiza una o varias de las actividades contempladas en esta Norma Internacional, tales como el muestreo o el diseño y desarrollo de nuevos métodos, los requisitos de los apartados correspondientes no se aplican.

\par \noindent
Esta Norma Internacional es para que la utilicen los laboratorios cuando desarrollan los sistemas de
gestión para sus actividades de la calidad, administrativas y técnicas. También puede ser utilizada por los clientes del laboratorio, las autoridades reglamentarias y los organismos de acreditación cuando confirman o reconocen la competencia de los laboratorios. Esta Norma Internacional no está destinada a ser utilizada
como la base para la certificación de los laboratorios.

\par \noindent
El cumplimiento de los requisitos reglamentarios y de seguridad relacionados con el funcionamiento de
los laboratorios no está cubierto por esta Norma Internacional.\cite{iso17025}

\subsubsection{Requisitos relativos a la gestión}

\begin{itemize}
	\item Control de Registros: El laboratorio debe establecer y mantener procedimientos para la identificación, la recopilación,
	la codificación, el acceso, el archivo, el almacenamiento, el mantenimiento y la disposición de los registros de
	la calidad y los registros técnicos. Los registros de la calidad deben incluir los informes de las auditorías
	internas y de las revisiones por la dirección, así como los registros de las acciones correctivas y preventivas.
	
	\par \noindent
	Todos los registros deben ser legibles y se deben almacenar y conservar de modo que sean
	fácilmente recuperables en instalaciones que les provean un ambiente adecuado para prevenir los daños, el deterioro y las pérdidas. Se debe establecer el tiempo de retención de los registros. Los registros se pueden presentar sobre cualquier tipo de soporte, tal como papel o soporte informático.
	
	\par \noindent
	Todos los registros deben ser conservados en sitio seguro y en confidencialidad.
	
	\par \noindent
	El laboratorio debe tener procedimientos para proteger y salvaguardar los registros almacenados
	electrónicamente y para prevenir el acceso no autorizado o la modificación de dichos registros.
	
	\item Registros Técnicos: El laboratorio debe conservar, por un período determinado, los registros de las observaciones
	originales, de los datos derivados y de información suficiente para establecer un protocolo de control, los
	registros de calibración, los registros del personal y una copia de cada informe de ensayos o certificado de
	calibración emitido. Los registros correspondientes a cada ensayo o calibración deben contener suficiente
	información para facilitar, cuando sea posible, la identificación de los factores que afectan a la incertidumbre y
	posibilitar que el ensayo o la calibración sea repetido bajo condiciones lo más cercanas posible a las
	originales. Los registros deben incluir la identidad del personal responsable del muestreo, de la realización de
	cada ensayo o calibración y de la verificación de los resultados.
	
	\par \noindent
	Las observaciones, los datos y los cálculos se deben registrar en el momento de hacerlos y
	deben poder ser relacionados con la operación en cuestión.
	
	\par \noindent
	Cuando ocurran errores en los registros, cada error debe ser tachado, no debe ser borrado,
	hecho ilegible ni eliminado, y el valor correcto debe ser escrito al margen. Todas estas alteraciones a los
	registros deben ser firmadas o visadas por la persona que hace la corrección. En el caso de los registros
	guardados electrónicamente, se deben tomar medidas similares para evitar pérdida o cambio de los datos
	originales.\cite{iso17025}
\end{itemize}

\subsubsection{Requisitos técnicos}
Muchos factores determinan la exactitud y la confiabilidad de los ensayos o de las calibraciones
realizados por un laboratorio.\cite{iso17025} Estos factores incluyen elementos provenientes:

\begin{itemize}
	\item Factores humanos: La dirección del laboratorio debe asegurar la competencia de todos los que operan equipos
	específicos, realizan ensayos o calibraciones, evalúan los resultados y firman los informes de ensayos y los
	certificados de calibración. Cuando emplea personal en formación, debe proveer una supervisión apropiada.
	El personal que realiza tareas específicas debe estar calificado sobre la base de una educación, una
	formación, una experiencia apropiadas y de habilidades demostradas, según sea requerido.
	
	\item Instalaciones y condiciones ambientales: Las instalaciones de ensayos o de calibraciones del laboratorio, incluidas, pero no de manera
	excluyente, las fuentes de energía, la iluminación y las condiciones ambientales, deben facilitar la realización
	correcta de los ensayos o de las calibraciones.
	
	\par \noindent
	El laboratorio debe asegurarse de que las condiciones ambientales no invaliden los resultados ni
	comprometan la calidad requerida de las mediciones. Se deben tomar precauciones especiales cuando el
	muestreo y los ensayos o las calibraciones se realicen en sitios distintos de la instalación permanente del
	laboratorio. Los requisitos técnicos para las instalaciones y las condiciones ambientales que puedan afectar a
	los resultados de los ensayos y de las calibraciones deben estar documentados.
	
	\par \noindent
	El laboratorio debe realizar el seguimiento, controlar y registrar las condiciones ambientales según lo
	requieran las especificaciones, métodos y procedimientos correspondientes, o cuando éstas puedan influir en
	la calidad de los resultados. Se debe prestar especial atención, por ejemplo, a la esterilidad biológica, el polvo,
	la interferencia electromagnética, la radiación, la humedad, el suministro eléctrico, la temperatura y a los
	niveles de ruido y vibración, en función de las actividades técnicas en cuestión. Cuando las condiciones
	ambientales comprometan los resultados de los ensayos o de las calibraciones, éstos se deben interrumpir.
	
	\item Métodos de ensayo y calibración, y de la validación de los métodos: El laboratorio debe aplicar métodos y procedimientos apropiados para todos los ensayos o las calibraciones dentro de su alcance. Estos incluyen el muestreo, la manipulación, el transporte, el almacenamiento y la preparación de los ítems a ensayar o a calibrar y, cuando corresponda, la estimación de la incertidumbre de la medición así como técnicas estadísticas para el análisis de los datos de los ensayos o de las calibraciones.
	
	\par \noindent
	El laboratorio debe tener instrucciones para el uso y el funcionamiento de todo el equipamiento pertinente, y
	para la manipulación y la preparación de los ítems a ensayar o a calibrar, o ambos, cuando la ausencia de
	tales instrucciones pudieran comprometer los resultados de los ensayos o de las calibraciones. Todas las
	instrucciones, normas, manuales y datos de referencia correspondientes al trabajo del laboratorio se deben
	mantener actualizados y deben estar fácilmente disponibles para el personal. 
	
	\item Equipos: El laboratorio debe estar provisto con todos los equipos para el muestreo, la medición y el ensayo,
	requeridos para la correcta ejecución de los ensayos o de las calibraciones (incluido el muestreo, la
	preparación de los ítems de ensayo o de calibración y el procesamiento y análisis de los datos de ensayo o
	de calibración). En aquellos casos en los que el laboratorio necesite utilizar equipos que estén fuera de su
	control permanente, debe asegurarse de que se cumplan los requisitos de esta Norma Internacional.
	
	\par \noindent
	Los equipos y su software utilizado para los ensayos, las calibraciones y el muestreo deben permitir
	lograr la exactitud requerida y deben cumplir con las especificaciones pertinentes para los ensayos o las
	calibraciones concernientes. Se deben establecer programas de calibración para las magnitudes o los valores
	esenciales de los instrumentos cuando dichas propiedades afecten significativamente a los resultados. Antes
	de poner en servicio un equipo (incluido el utilizado para el muestreo) se debe calibrar o verificar con el fin de
	asegurar que responde a las exigencias especificadas del laboratorio y cumple las especificaciones
	normalizadas pertinentes. El equipo debe ser verificado o calibrado antes de su uso.
	
	\item Trazabilidad de las mediciones: Todos los equipos utilizados para los ensayos o las calibraciones, incluidos los equipos para mediciones
	auxiliares (por ejemplo, de las condiciones ambientales) que tengan un efecto significativo en la exactitud o
	en la validez del resultado del ensayo, de la calibración o del muestreo, deben ser calibrados antes de ser
	puestos en servicio. El laboratorio debe establecer un programa y un procedimiento para la calibración de sus
	equipos.
	
\end{itemize}

\par \noindent
El grado con el que los factores contribuyen a la incertidumbre total de la medición difiere
considerablemente según los ensayos (y tipos de ensayos) y calibraciones (y tipos de calibraciones). El
laboratorio debe tener en cuenta estos factores al desarrollar los métodos y procedimientos de ensayo y de
calibración, en la formación y la calificación del personal, así como en la selección y la calibración de los
equipos utilizados.

\subsubsection{Aseguramiento de la calidad de los resultados de ensayo y calibración}

\par 
El laboratorio debe tener procedimientos de control de la calidad para realizar el seguimiento de la
validez de los ensayos y las calibraciones llevados a cabo. Los datos resultantes deben ser registrados en
forma tal que se puedan detectar las tendencias y, cuando sea posible, se deben aplicar técnicas estadísticas
para la revisión de los resultados. Dicho seguimiento debe ser planificado y revisado y puede incluir, entre
otros, los elementos siguientes:\cite{iso17025}

\begin{itemize}
\item el uso regular de materiales de referencia certificados o un control de la calidad interno utilizando
materiales de referencia secundarios;

\item la participación en comparaciones interlaboratorios o programas de ensayos de aptitud;

\item la repetición de ensayos o calibraciones utilizando el mismo método o métodos diferentes;

\item la repetición del ensayo o de la calibración de los objetos retenidos;

\item la correlación de los resultados para diferentes características de un ítem.
\end{itemize}

\par \noindent
Los datos de control de la calidad deben ser analizados y, si no satisfacen los criterios predefinidos,
se deben tomar las acciones planificadas para corregir el problema y evitar consignar resultados incorrectos.

\subsubsection{Informes de ensayos}
\par 
Los informes de ensayos deben incluir, en los casos en que sea necesario para la interpretación de los resultados de los ensayos, lo siguiente:\cite{iso17025}

\begin{itemize}
	\item las desviaciones, adiciones o exclusiones del método de ensayo e información sobre condiciones de
	ensayo específicas, tales como las condiciones ambientales;
	
	\item cuando corresponda, una declaración sobre el cumplimiento o no cumplimiento con los requisitos o las
	especificaciones;
	
	\item cuando sea aplicable, una declaración sobre la incertidumbre de medición estimada; la información
	sobre la incertidumbre es necesaria en los informes de ensayo cuando sea pertinente para la validez o
	aplicación de los resultados de los ensayos, cuando así lo requieran las instrucciones del cliente, o
	cuando la incertidumbre afecte al cumplimiento con los límites de una especificación;
	
	\item cuando sea apropiado y necesario, las opiniones e interpretaciones;
	
	\item la información adicional que pueda ser requerida por métodos específicos, clientes o grupos de clientes.
\end{itemize}



